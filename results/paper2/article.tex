\documentclass[12pt]{elsarticle}

\usepackage{IEEEtrantools}
\usepackage{amsmath}
\usepackage{mathrsfs}
% \usepackage{subcaption}
\usepackage{graphicx}
% \usepackage{subcaption}
\usepackage{setspace}
\usepackage{amsfonts}
\usepackage{subfig}
\usepackage{float}
\floatstyle{plaintop}
\restylefloat{table}
\usepackage[noheads,nomarkers,tablesfirst]{endfloat}
\usepackage[utf8]{inputenc}
\usepackage{lineno}
\usepackage{fixltx2e}

\begin{document}

\doublespacing

\begin{frontmatter}

  \title{Ecological clustering of the Red Sea and parallel 1D-ecological simulations}

  \author[1]{Denis Dreano}

  \author[3]{George Triantafyllou}

  \author[2]{Bani Mallick}
  
  \author[1]{Ibrahim Hoteit\corref{cor1}}
  \ead{ibrahim.hoteit@kaust.edu.sa} \cortext[cor1]{Corresponding author}


  \address[1]{Computer, Electrical and Mathematical Sciences and Engineering Division, King Abdullah University of Science and Technology}

  \address[2]{Department of Statistics, Texas A\&M University}

  \address[3]{Hellenic Center for Marine Research}

  \begin{abstract}
  Abstract
  \end{abstract}

\end{frontmatter}

\linenumbers


\section{Introduction}

3D ecological models are expensive to run. Can we divide the Red Sea
into regions and have 1D models running in each of them in parallel?

In this article we cluster the Red Sea in 3 different eco-regions
using automatic unsupervised learning algorithms. We then run an 
assimilative 1D ecological model on each of the region and analyze 
the results.

\section{Data}

\subsection{Chlorophyll data}

We use CCI monthly and 8-days CHL data.

\subsection{DINEOF}

CCI data present missing data, in particular, in the southern Red Sea
during summer. In order to have a complete dataset on which can apply a 
clustering algorithm, we use DINEOF, a data filling algorithm. The Chl
data is averaged over each region to give a data time-series for each of
them.

\subsection{Clustering}

We use clustering algorithm to divide the Red Sea into regions with
similar behavior. We tried K-means and Gaussian Mixture Model, a generalization
of the former.GMM was found to give better results.

\section{Model and Assimilation}

\subsection{1D-ERSEM model}

We use a 1D coupled ERSEM model. The physical forcing comes from
a 3D circulation simulation of the Red Sea [Yao 2014]. The ecological
models are initialized with the results of the 3D Red Sea ecology 
simulation [Triantfyllou2013].

\subsection{Data Assimilation}

To improve the results of the simulation we use the hybrid-SEIK
assimilation scheme, detailed in this subsection.

\section{Results}

\subsection{Model evaluation}

Here, we compare the results of the free-run with the assimilated-run.
We show that we have a good prediction skill, and that the assimilation
improves the model.

\subsection{Analysis}

Here we look at the results and interpret them biologycally. Do we find
comparable results as Acker, Raitsos, Weiker, etc. What can we say about the
hypothesis that they made about he process that drive primary productivity in
the Red Sea.

\section{Conclusion}

Are several 1D paralled 1D models a good alternative to 3D simulations?

What did we learn about the Red Sea ecology?

Future works?

\section*{Acknowledgment}

The research reported in this publication was supported by King Abdullah
University of Science and Technology (KAUST).

\section{Bibliography}

\bibliographystyle{elsarticle-num} \bibliography{bibtex}

\end{document}