\newcommand{\cov}{\text{cov}}

\newcommand{\0}{\mathbf{0}}

\newcommand{\C}{\mathbf{C}}
\newcommand{\X}{\mathbf{X}}
\newcommand{\U}{\mathbf{U}}
\renewcommand{\V}{\mathbf{V}}
\newcommand{\M}{\mathbf{M}}
\newcommand{\Q}{\mathbf{Q}}
\renewcommand{\H}{\mathbf{H}}
\newcommand{\I}{\mathbf{I}}
\newcommand{\R}{\mathbf{R}}
\renewcommand{\P}{\mathbf{P}}
\renewcommand{\L}{\mathbf{L}}

\renewcommand{\a}{\mathbf{a}}
\newcommand{\h}{\mathbf{h}}
\renewcommand{\r}{\mathbf{r}}
\newcommand{\x}{\mathbf{x}}
\renewcommand{\k}{\mathbf{k}}
\newcommand{\y}{\mathbf{y}}
\newcommand{\z}{\mathbf{z}}

\newcommand{\SSigma}{\mathbf{\Sigma}}
\newcommand{\GGamma}{\mathbf{\Gamma}}

\newcommand{\ttheta}{\boldsymbol\theta}
\newcommand{\eeta}{\boldsymbol\eta}
\newcommand{\vvarepsilon}{\boldsymbol\varepsilon}
\newcommand{\xxi}{\boldsymbol\xi}
\newcommand{\mmu}{\boldsymbol\mu}

\section{Data}

\subsection{CCI chlorophyll data}

\paragraph{We use CCI chlorophyll data because it has more coverage.}

\begin{itemize}
	\item Single satellite CHL data products have a lot of missing data
especially during summer in the South
  \item CCI data, merges three different sensors and uses the POLYMER
algorithm.
  \item As a result the coverage increases dramatically.
  \item This is the first dataset that has significant coverage in
the southern Red Sea, and that is why we will use it.
  \item We use 4km resolution L3 CHL product between such and such
coordinates.
  \item We use weekly data for the clustering and 8days data for
the assimilation
\end{itemize}

\paragraph{With a quick look at the data this is what we see...}

\begin{itemize}
  \item plot coverage
  \item plot average chlorophyll
  \item plot seasonal chlorophyll
\end{itemize}

\subsection{DINEOF}

\paragraph{There are still missing data in CCI, so we use DINEOF for
data filling because...}

\begin{itemize}
  \item DINEOF is an EOF based non parametric data filling methods
  \item introduced orignally by \citet{Beckers2003}
  \item Has been applied to geoscience datasets, in particular to
chl datasets
  \item Shown in \citet{Taylor2013} to be more efficient that its
competition.
\end{itemize}

\paragraph{More or less this is the way DINEOF works...}

\begin{itemize}
  \item Describe here the algo
\end{itemize}

\paragraph{This is how we applied DINEOF...}

\begin{itemize}
  \item Choices of parameters and cross validation method and why we
made these choices.
\end{itemize}

\paragraph{Present and discuss the results of DINEOF}

\begin{itemize}
  \item Show minimization of error plot.
  \item Show the filling of a region.
\end{itemize}

\subsection{Clustering}

\paragraph{To identify the ecological regions, we cluster the Red Sea using
clustering algorithms on the CHL data. We chose GMM because...}

\begin{itemize}
  \item There are many clustering algorithms on the market: for example...
  \item Discuss advantages and inconvenients of some of them (find reference)
  \item Finally we tried some of them and found that GMM was given better
results, in comparison to what we expected.
\end{itemize}

\paragraph{This is more or less the way GMM works...}

\begin{itemize}
  \item Describe how GMM algo works
\end{itemize}

\paragraph{This is how we used it...}

\begin{itemize}
  \item We wanted 3 broad regions in the northern, central and southern Red Sea.
  \item We ran GMM with k varying from 3 to 7.
  \item We looked at some of this tests to see if the clusters where good.
  \item At the end we settled with k=? because it was good for our purposes and
the regions where closed to to what \citet{Raitsos2013} found.
\end{itemize}

\paragraph{This is what we got...}

\begin{itemize}
  \item Show plot of clusters
  \item Comment on clusters, and what was found by \citet{Raitsos2013}.
\end{itemize}
