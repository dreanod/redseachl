\section{Introduction}

\paragraph{What are marine ecosystem models}
ERSEM, NPZ.

\paragraph{Marine ecosystem models have scientific and practical applications}
Forecasting, harmful algae blooms, scientific inquiry when not enough data.

\paragraph{Complex 3D ecological models have limitations}
Indetermination and computational cost.

\paragraph{Assimilation in necessary but increases the computational cost}
Ecological systems are chaotic system that require assimilation to
maintain a good prediction skill. But assimilation increases the computational
cost. Assimilation of eco models remains a challenge.

\paragraph{The hybrid-SEIK}
Has never been use for eco models before.

\paragraph{Ocean colour data has been assimilated a lot to eco models}
chlorophyll is a green colored pigment critical for photosynthesis and found
in plants and algae \citep{Pal2014}. Give a detectable green coloration
to the water when phytoplankton is present \citep{Robinson2010}.
Remotely sensed ocean colour data give highly available data with 
high coverage both in time and space \citep{Robinson2010}. Ocean colour
data products like chl are very good proxy for phytoplankton concentration.

\paragraph{But has some limitations}
CHL dataset have missing data because of clouds. Which is a big problem in 
the southern Red Sea in summer where the coverage is almost zero
\cite{Racault}. Moreover the chl is difficult to estimate in case II
optically complex waters, especially near the caosts. It also particularly
affects the southern Red Sea which is very shallow.

\paragraph{Solutions to missing data}
One of the most popular approach for data filling of remotely-sensed 
chlorophyll data is DINEOF introduced by \citet{Beckers2003}.

\paragraph{The Red Sea is a relatively unexplored sea}
Not a lot of studies, not a lot of in situ data.
Must use models and remotely-sensed data.
RS is TTS with strong stratification that limits vertical diffusion of nutrients.
\citep{Mann2006}.
Other than the gulf of Aden \citep{Yao2015}, 
the RS has no known significant input of nutrients and
is oligotrophic \citep{Raitsos2013, Weikert1987}.
Red Sea has a rich ecosystem and unique ecosystem that has adapted its
extreme environment \citep{Raitsos2011}.
RS relatively well preserved but increasingly fragilized by human activities.
Sharp increase of temperature in the past decade threatend the RS environment.
\citep{Raitsos2011}.

\paragraph{Major biological patterns}
Current hypotheses about primary production in Red Sea.
There is a lack of missing data therefore the large scale ecological 
dynamics is poorly known \citep{Raitsos2013, Triantafyllou2014}.
The role of aerosol deposition could be important but has not been investigated
yet \citep{Raitsos2013}.
Eddies are believed to play important role \citep{Raitsos2013, Zhan2014}.
Chlorophyll increase from north to south \citep{Raitsos2013}.
Secondary summer bloom (but not in NRS) \citep{Raitsos2013}.
Strong Interannual variability \citep{Raitsos2013}.
Exchange of water with GOA is a major driver of productivity for the 
whole Red Sea \citep{Triantafyllou2014}.
SRS winter bloom attributed to wind-driven intrusion \citep{Raitsos2013}.
Deep convection in winter plays a big role in the northern Red Sea
\citep{Raitsos2013}.
Red Sea circulation impacted by eddies that could impact productivity
\citep{Zhai2013}. Central red Sea anti cyclonic eddy is believed to
control June peak \citep{Raitsos2013}.
Climate mode indices have impact ton the Red Sea \citep{Raitsos2015}.


\paragraph{Objective:}
3D ecological models are expensive to run. Can we divide the Red Sea into
regions and have 1D models running in each of them in parallel?
In this article we cluster the Red Sea in 3 different eco-regions using
automatic unsupervised learning algorithms. We then run an assimilative 1D
ecological model on each of the region and analyze the results.

\paragraph{Introduce chapters}

\begin{quotation} 

Typical tropical seas (TTS), like the Red Sea, are characterized by a highly
stratified structure, where warm nutrient-depleted surface water is separated
from the cold nutrient-rich deep water by a steep gradient of temperature zone
called pycnocline. The pycnocline acts as barrier that limits the upward
nutrients flow \citep{Mann2006}. As a result, TTS are oligotrophic and have low
chlorophyll concentrations. Until recently, marine biologists believed that
tropical and subtropical seas have therefore a very low productivity. However,
recent investigations have contested this idea, suggesting that different
upwelling mechanisms (winter deep mixing, storms, eddies, etc) exist, which
bring new nutrients to the surface water \citep{Mann2006}.

Despite being an oligotrophic and challenging environment for marine life, the
Red Sea presents a surprisingly rich and diverse ecosystem \citep{Raitsos2011},
and a very well developed coral reef system \citep{Racault}. The source of
nutrient for sustaining such a developed ecosystem is not well understood yet,
but the exchange with the open ocean, the atmospheric depositions and transport
through the mesoscale eddies are believed to play an important role
\citep{Raitsos2013, Zhan2014}.

Although the Red Sea environment is still relatively well preserved, it is
increasingly stressed by human activities. The continuous urbanization and
fishing activity contribute to the fragilization of this unique ecosystem
\citep{Acker2008}. An abrupt increase of temperature has further occurred in
the last decade, which may threaten the fragile coral reef system
\citep{Raitsos2011}.

Because of the lack of in-situ data, the large-scale phytoplankton dynamics of
the Red Sea remain largely unknown \citep{Raitsos2013, Triantafyllou2014}.
However, in recent studies, remotely-sensed data and computer simulations have
been used to improve our knowledge of the ecology of this region. The Red Sea
is deficient in major nutrients \citep{Weikert1987}, and the only significant
input of water comes from the Gulf of Aden \citep{Yao2015}. This explains a
general pattern of chlorophyll concentration increase from north to south
\citep{Raitsos2013}, with the lowest concentration found in the northern
central Red Sea. This pattern can be seen in figure \ref{meanchl}.  The Red Sea
also displays a distinct seasonality, with a peak in concentration during the
winter.  A weak summer peak is also observed around July, everywhere except in
the northernmost region \citep{Raitsos2013}. Despite this regularity, a strong
interannual variability is observed, with blooms that can reach mesotrophic
concentration levels \citep{Raitsos2013}. According to
\citet{Triantafyllou2014}, the variations in the Red Sea ecology are mainly
driven by physical circulation. In the rest of this section, we explore some of
the mechanisms that have been linked to the major features of chlorophyll
concentration.

%\begin{figure}[h] \centering
%\includegraphics[scale=.15]{figures/chl_average.png} \caption{Average
%chlorophyll concentration from CCI data} \label{meanchl} \end{figure}

The exchange of water with the nutrient-rich Gulf of Aden is a major driving
mechanism for the whole Red Sea \citep{Triantafyllou2014}. It is the most
important source of nutrients. The maximum chlorophyll concentration observed
in the southern Red Sea during winter is attributed to wind-driven water
intrusion \citep{Raitsos2013}. In Summer, this exchange of water is believed to
be the only significant source of nutrients for the whole Red Sea. The
influence of the water intrusion weakens as the latitude increases, explaining
the low concentration in the northern half of the Red Sea \citep{Raitsos2013}.

Deep convection also plays an important role in allowing nutrient-rich deep
water to mix with water of the euphotic zone. Vertical mixing is the most
vigorous in the northern extremity of the Red Sea during the winter. This
explains its higher chlorophyll concentration compared to the north-central Red
Sea, a region of weak mixing \citep{Raitsos2013}. The northern Red Sea mixing
is believed to be driven by wind \citep{Raitsos2013}.

The Red Sea circulation is strongly influenced by mesoscale eddies
\citep{Yao2014, Yao2014b, Zhan2014} that could impact primary production
\citep{Zhai2013}.  In particular, the anti-cyclonic eddy in the central Red Sea
is believed to control the June concentration peak and the summer productivity
of this region, by transporting nutrients and/or phytoplankton from the
adjacent coral reefs \citep{Raitsos2013}. In the northern Red Sea, a cold-core
eddy plays a role in enhancing the vertical mixing in this region
\citep{Raitsos2013}.

Aerial depositions of dust could also be an important input of nutrients for
the Red Sea, but it has been largely left unexplored \citep{Triantafyllou2014}.
\citet{Raitsos2013} noticed for example that sand storms in the Red Sea most
frequently happen in June and July, which coincides with the summer chlorophyll
peak. Finally, climate mode indices have been shown to be strongly correlated
with air-sea heat exchanges in the Red Sea \citep{Abualnaja2015}, and might
therefore influence its biology. This has been recently confirmed by
\citet{Raitsos2015}, who have shown that El Nino has a positive impact on the
chlorophyll concentration, by strenghtening the wind transporting nutrients
into the Red Sea from the Gulf of Aden.  

\end{quotation}

\begin{quotation} 

Chlorophyll is a molecule present in algae, phytoplankton and
plants that is critical for photosynthesis. It is a poor absorber of green
light, and is responsible for the coloration of plants \citep{Pal2014}. When
phytoplankton are present in high concentrations, the water also takes a
detectable green coloration (it can also take a red or blue coloration
depending on the type of dominating phytoplankton) \citep{Robinson2010}. This
offers an efficient way to observe the phytoplankton concentration from space.

In-situ measurement of chlorophyll concentration can be gathered through
scientific cruises, buoy stations or gliders (unmanned submarines). These
methods are expensive to deploy and therefore generally have limited temporal
and spatial coverage \citep{Robinson2010}. Political issues, as in the Red Sea,
as well as security issues, as in the Arabian Sea, set also barriers to in-situ
measurements.

Satellite measurements of chlorophyll provide excellent proxies for
phytoplankton concentrations with a good temporal and spatial coverage
\citep{Robinson2010}. The SeaWIFS, MODIS and MERIS missions have provided an
uninterrupted coverage of the world since 1997. High-resolution maps of daily
chlorophyll concentration are freely accessible to the scientific community
\citep{McClain2009}. Despite some limitations, like missing data due to cloud
coverage and sunglint, or problematic values in coastal areas, remotely-sensed
chlorophyll concentrations are intensively used by the scientific community. In
regions, like in the Red Sea, where little in-situ measurements are available
\citep{Raitsos2013, Brewin2013}, these constitute the most important data
source.

%\verbatim{\subsection{Limitation of Remotely-Sensed Chlorophyll Data in the Red Sea}}

The quality of remotely-sensed chlorophyll data products such as MODIS and
SeaWiFS in the Red Sea is comparable with that of the rest of the world for
case I waters (open sea) \citep{Brewin2013}. However, the data contains a large
amount of missing values because of persistent clouds, sun-glint and sensor
saturation \citep{Racault}. This problem is particularly acute during the
summer in the southern Red Sea where the data coverage is almost null
\citep{Racault}, as shown in figure \ref{misval_modis}.

Chlorophyll concentration estimation in optically complex case II waters is a
recurrent problem in this remotely-sensed data that particularly affects the
southern Red Sea.  In this region, the remotely sensed chlorophyll data could
be overestimated \citep{Raitsos2013}. However, all high values are not
necessarily bad, as highly productive coral reefs are also present in this
region \citep{Raitsos2013}.  However, these values have not been validated yet,
due to the lack of in situ measurements \citep{Raitsos2013}.

%\begin{figure}[h] \centering
%\includegraphics[scale=.15]{figures/modis_missing_values_summer.png}
%\caption{Percentage of missing values in the MODIS chlorophyll dataset}
%\label{misval_modis} \end{figure}

One solution to missing and bad values is to use a data filling algorithm, of
which one of the most popular is the Date INterpolating Empirical Orthogonal
Functions (DINEOF). It is an EOF based data filling approach introduced by
\citet{Beckers2003}. In \citet{Sirjacobs2011}, it has been employed to fill
chlorophyll data with 70\% of missing values.  \citet{Taylor2013} has compared
DINEOF with other EOF-based reconstruction algorithms, suggesting that the
former is the best method for data filling.  DINEOF has been employed in
several other chlorophyll studies \citep{Miles2010, Waite2013}. It has also
been used for multivariate reconstruction of SST fields using chlorophyll data
in \citet{Alvera2007}. 

The OC-CCI is a new chlorophyll data product that considerably increases the
Red Sea coverage. It merges the data from sensors SeaWiFS, MODIS and MERIS.
Overall, it achieves a 75-80\% coverage in the entire Red Sea basin against
50-65\% for a single sensor \citep{Racault}. During the summer, the improvement
is dramatic, as shown in figure \ref{misval_cci}. This is mostly due to the use
of the POLYMER algorithm \citep{Steinmetz2011} that allows to exploit MERIS
data collected during hazy conditions. However, this new dataset has not been
fully explored to revisit the assumptions made on the large-scale Red Sea
phytoplankton productivity.

%\begin{figure}[h] \centering
%\includegraphics[scale=.15]{figures/cci_missing_values_summer.png}
%\caption{Percentage of missing values in the CCI chlorophyll dataset}
%\label{misval_cci} \end{figure}

%\verbatim{\section{Modeling and Forecasting Chlorophyll: Data-Driven and Physics-Driven
%Approaches, and Applications}}

%\verbatim{\subsection{Why Modeling Chlorophyll?}}

Models could be useful to identify causes behind the chlorophyll patterns we
observe in the Red Sea. Many hypotheses have been made about the drivers of
chlorophyll concentration in this regions, but some of them have not been yet
investigated through models. The role played by the exchange of water with the
Gulf of Aden and winter overturning in the northern Red Sea have been
successfully modeled with a 3D coupled ecological model
\citep{Triantafyllou2014}. However, the interaction between the open sea and
coral reefs, and the role of atmospheric depositions have not been investigated
yet. Models, can also be helpful for understanding governing dynamics affecting
the chlorophyll concentration. In particular, the interaction between the
productivity level of the different regions of the Red Sea is yet to be
explored.

Model predictions of chlorophyll concentration also have practical
applications. Phytoplankton blooms can be harmful to humans and marine life and
are closely monitored in many regions of the world \citep{Pettersson2013}. In
the Red Sea, where tourism and aquaculture are developing, it is likely to
become a concern too. Phytoplankton is also directly, and indirectly through
zooplankton, the cause of microfouling that affects desalination plants. In
2008-2009, a red tide forced the shutdown of desalination plants along the Gulf
of Oman and the Arabian Gulf \citep{Richlen2010}.


%\verbatim{\subsection{Deterministic Models}}

%\verbatim{\subsubsection{Ecological Models}}

There is a rich literature on the modeling of marine ecosystem using
differential equations (see \citet{Fennel2004} for an introduction In these
models the interactions of complex physical, chemical and biological processes
are modeled by differential equations that represent the flow of carbon,
nitrogen, phosphate and silicon. The biota is divided into trophic levels, and
can be further divided by feeding methods and size classes
\citep{Triantafyllou2014}.

Ecological deterministic models in use vary widely in diversity depending on on
the number of state parameters and interactions represented.  They can be as
simple as the nutrient-phytoplankton-zooplankton (NPZ) model
\citep{Anderson2005} that only has three variables representing two trophic
levels and nitrate, or as complex as the European regional seas ecosystem model
(ERSEM) that has dozens of variables \citep{Baretta1995}. NPZ models are
extensively used because of its simplicity and its capacity to model the the
large-scale features of marine ecosystems \citep{Anderson2005}.  ERSEM has been
use in many studies. It has recently been coupled to the MITgcm circulation
model used to simulate the Red Sea ecology \citep{Triantafyllou2014}. However
the complexity of these models makes them difficult to parametrize if not
enough data are available, which is usually the case \citep{Anderson2005}.

%\verbatim{\subsubsection{Data Assimilation}}

Data assimilation is used to improve the simulations of ecological dynamical
models and enhance their forecasting capabilities by constraining their
predictions with available observations \cite{Edwards2015}. Such prediction
capabilities are deployed in operational expert systems, for example to study
the impact of human activities on the ecosystem of the Gulf of Pagasitikos
\citep{Korres2012}. The deployment of a similar forecasting system in the Red
Sea is currently under development \citep{Triantafyllou2014}. Hindcasting, the
estimation of unobserved variables, is another application of assimilation
scheme. \citet{Ciavatta2011}  showed that they could improve the seasonal and
annual hindcast of non-assimilated biogeochemical properties in the shelf area
of Western English Channel. Data assimilation can also be used for reanalysis,
to provide estimates of past years biogeochemical variables
\citep{Fontana2013}. 

In the marine ecology modeling community, three assimilation schemes have been
widely used: the Ensemble Kalman filters (EnKF), the Singular Evolutive
Extended Kalman filter (SEEK), and its ensemble variant, the Singular Evolutive
Interpolated Kalman filter (SEIK). The stochastic EnKF, a Monte-Carlo
approximation of the Kalman Filter, has been used in \citet{Ciavatta2011,
Ciavatta2014}. This scheme may however suffer from sampling errors when the
ensemble size is smaller than the number of observations, as is usually the
case when assimilating remotely-sensed data. SEEK is a reduced-rank variant of
the Extended Kalman filter (EK). It was introduced to run efficiently when the
state dimension is very large, as is the case in ocean applications. It is
based  on the projection of the error covariance onto a low dimensional space.
SEEK has a long history in data assimilation for marine ecology models and is
still extensively used in recent studies \citep{Fontana2013, Korres2012,
Butenschon2012}.  SEIK is an ensemble variant of the SEEK  and a deterministic
version of the EnKF that do not suffer from observations sampling errors, as it
updates the filter and forecast exactly as in the Kalman filter, but requires a
resampling step to generate a new ensemble for the next forecast step. SEIK has
been used by \citep{Triantafyllou2013, Korres2012}. \citet{Korres2012} shows
that SEIK and SEEK are both comparably robust methods for highly nonlinear
systems. \citet{Hoteit2005} has shown that SEIK outperforms SEEK when using a
high-resolution non linear model.

Ecological models are challenging applications for state of the art data
assimilation schemes \citep{Edwards2015}. First, biogeochemical variables are
usually positive concentration, whereas Kalman filters expect Gaussian
variables, and log-transformation may fail at avoiding this issue
\citep{Ciavatta2011}. In an attempt to mitigate this problem,
\citet{Fontana2013} has introduced Gaussian anamorphosis transformations.
Second, ecological blooms are intermittent and highly nonlinear, conditions
that are challenging for Kalman filter-based assimilation schemes
\citep{Triantafyllou2013, Korres2012}. Third, SEIK, EnKF and SEEK project the
error covariance onto some subspace, resulting in an underestimation of the
estimation error. \citet{Butenschon2012} studied different ways to propagate
the error covariance in order to alleviate this issue. Finally, the model error
statistics are required by Kalman-derived filters, but are difficult to
estimate. \citet{Triantafyllou2013} proposed to use the $H_\infty$ method with
SEIK in order alleviate this requirement.

Particle filters represent a class of data assimilation scheme that are not
derived from the Kalman filter, and do not make any linearity or Gaussianity
assumption \citep{Edwards2015}. They have been studied in the case of 0D and 1D
ecological models \citep{Edwards2015}.  Their application to 3D model is an
active field of research \citep{Edwards2015}.

\end{quotation}

