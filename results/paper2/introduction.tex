\section{Introduction}

\paragraph{3D marine ecological models are useful...}

\paragraph{...But expensive and difficult to run}

\paragraph{In this article we look at ways to simulate 3D ecosystems more
cheaply by running many parallel 1D regional models}

\paragraph{We are going to test that idea on the Red Sea because...}

\paragraph{We will also the hybrid-SEIK data assimilation scheme, because...}

\paragraph{We will assimilate Chl data even if it is imperfect because it is
the best available data for the Red Sea}

\paragraph{What we are going to do in this paper step by step}

\paragraph{What is new in this paper and why}

\paragraph{Introduce sections}


% 
% \paragraph{What are marine ecosystem models}
% ERSEM, NPZ.
% 
% \paragraph{Marine ecosystem models have scientific and practical applications}
% Forecasting, harmful algae blooms, scientific inquiry when not enough data.
% 
% \paragraph{Complex 3D ecological models have limitations}
% Indetermination and computational cost.
% 
% \paragraph{Assimilation in necessary but increases the computational cost}
% Ecological systems are chaotic system that require assimilation to
% maintain a good prediction skill. But assimilation increases the computational
% cost. Assimilation of eco models remains a challenge.
% 
% \paragraph{The hybrid-SEIK}
% Has never been use for eco models before.
% 
% \paragraph{Ocean colour data has been assimilated a lot to eco models}
% chlorophyll is a green colored pigment critical for photosynthesis and found
% in plants and algae \citep{Pal2014}. Give a detectable green coloration
% to the water when phytoplankton is present \citep{Robinson2010}.
% Remotely sensed ocean colour data give highly available data with 
% high coverage both in time and space \citep{Robinson2010}. Ocean colour
% data products like chl are very good proxy for phytoplankton concentration.
% 
% \paragraph{But has some limitations}
% CHL dataset have missing data because of clouds. Which is a big problem in 
% the southern Red Sea in summer where the coverage is almost zero
% \cite{Racault}. Moreover the chl is difficult to estimate in case II
% optically complex waters, especially near the caosts. It also particularly
% affects the southern Red Sea which is very shallow.
% 
% \paragraph{The Red Sea is a relatively unexplored sea}
% Not a lot of studies, not a lot of in situ data.
% Must use models and remotely-sensed data.
% RS is TTS with strong stratification that limits vertical diffusion of nutrients.
% \citep{Mann2006}.
% Other than the gulf of Aden \citep{Yao2015}, 
% the RS has no known significant input of nutrients and
% is oligotrophic \citep{Raitsos2013, Weikert1987}.
% Red Sea has a rich ecosystem and unique ecosystem that has adapted its
% extreme environment \citep{Raitsos2011}.
% RS relatively well preserved but increasingly fragilized by human activities.
% Sharp increase of temperature in the past decade threatend the RS environment.
% \citep{Raitsos2011}.
% 
% \paragraph{Major biological patterns}
% Current hypotheses about primary production in Red Sea.
% There is a lack of missing data therefore the large scale ecological 
% dynamics is poorly known \citep{Raitsos2013, Triantafyllou2014}.
% The role of aerosol deposition could be important but has not been investigated
% yet \citep{Raitsos2013}.
% Eddies are believed to play important role \citep{Raitsos2013, Zhan2014}.
% Chlorophyll increase from north to south \citep{Raitsos2013}.
% Secondary summer bloom (but not in NRS) \citep{Raitsos2013}.
% Strong Interannual variability \citep{Raitsos2013}.
% Exchange of water with GOA is a major driver of productivity for the 
% whole Red Sea \citep{Triantafyllou2014}.
% SRS winter bloom attributed to wind-driven intrusion \citep{Raitsos2013}.
% Deep convection in winter plays a big role in the northern Red Sea
% \citep{Raitsos2013}.
% Red Sea circulation impacted by eddies that could impact productivity
% \citep{Zhai2013}. Central red Sea anti cyclonic eddy is believed to
% control June peak \citep{Raitsos2013}.
% Climate mode indices have impact ton the Red Sea \citep{Raitsos2015}.
% 
% 
% \paragraph{Objective:}
% 3D ecological models are expensive to run. Can we divide the Red Sea into
% regions and have 1D models running in each of them in parallel?
% In this article we cluster the Red Sea in 3 different eco-regions using
% automatic unsupervised learning algorithms. We then run an assimilative 1D
% ecological model on each of the region and analyze the results.
% 
% \paragraph{Introduce chapters}
