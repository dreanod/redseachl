\section{Introduction}

\paragraph{Marine ecosystem forecasting is needed because...}

\begin{itemize}
  \item HAB affect public health, desalination and coastal economy
  \item predicting chlorophyll can help fisheries
  \item For research: better understand the large-scale ecosystem
  \item Especially usefull because we lack data about the subsurface
phenomena
\end{itemize}

\paragraph{Marine ecological models are one way to make such forecasting}

\begin{itemize}
  \item marine ecology models represent biogeochemical interactions as
differential equations.
  \item Can be as simple as NPZ or as complex and complete as ERSEM.
\end{itemize}

\paragraph{...But are expensive and difficult to tune, and subject to
various sources of uncertainties}

\begin{itemize}
  \item \citep{Anderson2005} lot of under-determination
  \item require coupling physics + biology
  \item Circulation models very expensive, because high-resolution grids
and number of involved PDEs.
\end{itemize}

\paragraph{Data assimilation and parameter estimation techniques are
used to improve and tune the models, but 3D models are still expensive
to run}

\paragraph{In this article we investigate ways to simulate and predict
3D ecosystems more cheaply by running many parallel 1D regional models}

\begin{itemize}
  \item Divide the domain into small regions with similar ecology using
advanced clustering techniques
  \item Build 1D model for each identified region
  \item Ensure parametrization is better for each region
  \item apply data assimilation techniques on the 1D models for efficient
calculations.
\end{itemize}

\paragraph{We are going to test that idea on the Red Sea because...}

\begin{itemize}
  \item Red Sea is an interesting environment: extreme temperatures and
salinity
  \item Very rich and preserved ecosystem
  \item Unexplored environment
  \item Lack of data: therefore developing models is important
\end{itemize}

\paragraph{We will use the SEIK data assimilation scheme, because...}

\begin{itemize}
  \item Assimilation constrains the model, reduces underdetermination,
and imporves forecasting
  \item Mitigates the fact that initial conditions, parameters and physics
are subject to uncertainties.
  \item SEIK is better than SEEK for strongly nonlinear models
\end{itemize}

\paragraph{We will assimilate Chl data because it is currently
the best available data for the Red Sea}

\begin{itemize}
  \item Chl data allows to observe large scale ecological patterns with
high spatial and temporal coverage.
  \item Compared with in situ data that are limited in time and space,
and expensive.
  \item However chl data suffers from missing values due clouds, aerosols, etc.
  \item Also bad values near the coast, case II waters
  \item Both problem particularly affect the southern Red Sea, that has
nearly no observation in the summer during some months.
\end{itemize}

\paragraph{What we are going to do in this paper step by step}

\begin{itemize}
  \item Fill the data with DINEOF
  \item Apply clustering to the Red Sea
  \item Implement 1D models and assimilation schemes
  \item Run models with assimilation
  \item Analyze results and compare to findings in previous studies
\end{itemize}

\paragraph{What is new in this paper and why}

\begin{itemize}
  \item We use CCI data: which is a new dataset, not fully exploited in
Red Sea.
  \item We do eco-region clustering for the first time in the Red Sea.
  \item We have assimilation of ecological model with hybrid-SEIK published
for the first time.
  \item We improve our understanding of the Red Sea ecology in its different
parts.
\end{itemize}

\paragraph{Introduce sections}

