\section{Introduction}

\paragraph{3D marine ecological models...}

\begin{itemize}
  \item marine ecology models represent biogeochemical interactions as
differential equations.
  \item Can be as minimal as NPZ or as complete as NPZ
\end{itemize}

\paragraph{...are useful because...}

\begin{itemize}
  \item HAB affect public health, desalination and coastal economy
  \item predicting chlorophyll can help fisheries
  \item For research: better understand the large-scale ecosystem
  \item Especially usefull because we lack data about the subsurface
phenomena
\end{itemize}

\paragraph{...But expensive and difficult to run}

\begin{itemize}
  \item \citep{Anderson2005} lot of underdetermination
  \item circulation model very expensive, because very small grid
\end{itemize}

\paragraph{In this article we look at ways to simulate 3D ecosystems more
cheaply by running many parallel 1D regional models}

\begin{itemize}
  \item Divide the Red sea in small regions with similar ecology
  \item Reduces underdetermination
  \item Ensure parametrization is better for each region
\end{itemize}

\paragraph{We are going to test that idea on the Red Sea because...}

\begin{itemize}
  \item Red Sea is an interesting environment: extreme temperatures and
salinity
  \item Very rich and preserved ecosystem
  \item Unexplored environment
  \item Lack of data: therefore developing models is important
\end{itemize}

\paragraph{We will use the hybrid-SEIK data assimilation scheme, because...}

\begin{itemize}
  \item Assimilation constrains the model and reduces underdetermination
  \item Mitigates the fact that initial conditions are unknown
  \item SEIK is better than SEEK for strongly nonlinear models
  \item SEIK is better than EnKF when fewer observations than states
  \item hybridization reduces the ensemble size and the computational cost
\end{itemize}

\paragraph{We will assimilate Chl data even if it is imperfect because it is
the best available data for the Red Sea}

\begin{itemize}
  \item Chl data allows to observe large scale ecological patterns with
high spatial and temporal coverage.
  \item Compared with in situ data that are limited in time and space,
and expensive.
  \item However chl data suffers from missing values due clouds, aerosols, etc.
  \item Also bad values near the coast, case II waters
  \item Both problem particularly affect the southern Red Sea, that has
nearly no observation in the summer during some months.
  \item However as lack of in situ data, this is the best we have
currently in the Red Sea
\end{itemize}

\paragraph{What we are going to do in this paper step by step}

\begin{itemize}
  \item Fill the data with DINEOF
  \item Apply clustering to the Red Sea
  \item Implement 1D models
  \item Run models with assimilation
  \item Analyze results and compare to findings in previous studies
\end{itemize}

\paragraph{What is new in this paper and why}

\begin{itemize}
  \item We use CCI data: which is a new dataset, not fully exploited in
Red Sea.
  \item We do eco-region clustering for the first time in the Red Sea.
  \item We have assimilation of ecological model with hybrid-SEIK published
for the first time.
  \item We improve our understanding of the Red Sea ecology in its different
parts.
\end{itemize}

\paragraph{Introduce sections}

