\section{Model and Assimilation}

\subsection{1D-ERSEM model}

\paragraph{Description of ERSEM}

\begin{itemize}
  \item ERSEM develop originally for the northern Sea
  \item Complete ecology modeling
  \item Been applied in many ecosystem
  \item in particular used for the Red Sea simulation \citet{Triantafyllou2013}
  \item Very complex: many parameters and variables
\end{itemize}

\paragraph{Initialization/Parameters/Forcing}

\begin{itemize}
  \item We initialize with the values found by \citet{Triantafyllou2013}
  \item The nutrients are initialized using the values of WOA
  \item Parameters are chosen like this
  \item Forcing come from the simulation by \citet{Yao2014b, Yao2014}
\end{itemize}

\subsection{Data Assimilation}

\paragraph{We chose hybrid-SEIK DA scheme because...}

\begin{itemize}
  \item Data assimilation is necessary to improve the forecasting skill of
complex geophysical models
  \item It constrains models that are imperfect and whose parametrization is
difficult to do.
  \item SEEK has a long history in assimilation into ecological models.
  \item Its ensemble variant SEIK, has been shown to behave better for very
nonlinear systems
  \item However, SEIK requires to run the model many time in parallel.
  \item To reduce the ensemble size and improve the efficiency, we use the
hybrid formulation proposed in \citet{Hamill2011}.
\end{itemize}

\paragraph{Equations of hybrid-SEIK}

\begin{itemize}
  \item Show the steps of the hybrid-SEIK scheme
\end{itemize}

\paragraph{Parameters}

\begin{itemize}
  \item How we choose the scheme parameters and why.
\end{itemize}
