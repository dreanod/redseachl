
% Chapter 3 File

\chapter{Editings}
\label{chapter3}

\section{Task 1: Dataset building and Exploration}

Duration: 2 months (by December 2014)

Submission: Journal of Marine Systems

Collaborator: Dionysios Raitsos

\subsection{Motivation}

A preliminary task to data modeling, is the gathering, cleaning and exploration of the data. Given the complexity and the size (40 GB) of the data, this is not an easy task. This first data analysis, will reveal if enough data has been gathered to make meaningful forecast, and what accuracy we can expect from the models. This step will also provide information that will help in designing statistical models: most significant variables, differences between regions, relevant data transformation, etc. Finally, this step will identify patterns in the data that will be useful to qualitatively evaluate predictive models.

\subsection{Open Questions}

\begin{itemize}
\item Can we efficiently identify outliers in the chlorophyll values?
\item Is there a way to efficiently fill the missing values in the chlorophyll dataset?
\item Can the data help understanding the mechanisms behind extreme blooms in the Red Sea?
\item Can the hypothesizes about the dynamics behind the chlorophyll seasonal cycle be confirmed by the data?
\item Are there more blooms in the past years?
\end{itemize}

\subsection{Method and Work Done}

Identify data sources and load the data…………………………………………………60%
Clean the data and fill missing values (DINEOF)......................................................50%
Align and format the data in order to have a unique dataset…………………………...0%
Explore the dataset………………………………………………………………………..20%
Study the correlation between chlorophyll and other variables (Linear Regression, GAM, data transformations)
Select variables (Lasso, single variable regression, multistep regression)
Study the regional aggregation (ACF)
Explore spatiotemporal correlations (hovmoller plots, PCA, variograms)
Estimate the Bayes factor/ % of variance explained (k-nearest neighbors)
Expected Outcomes
A cleaned dataset that can be used in the following tasks
A comprehensive exploration of the available data for chlorophyll study in the Red Sea
A preliminary variable selection
A clear picture of the major spatio-temporal patterns in the data
A critical evaluation of the current hypothesis about the chlorophyll dynamics in the Red Sea


\section{Task 2:  Forecasting Chlorophyll Concentration in Regional Aggregates}

Duration: 2 months (by February 2015)

Submission: Progress in Oceanography

Collaborator: Dionysios Raitsos

\subsection{Motivation}

Chlorophyll data is very complex. It is therefore useful to first simplify it by aggregating it spatially. The space-time dynamics of the chlorophyll data reflects the highly nonlinear dynamics of the underlying physical, chemical and biological phenomenons. As shown by the north-south gradient and the seasonal behavior, the resulting space-time process is nonstationary in time and in space. The high-dimensionality in space can be reduced by considering a regional aggregation of the results. This would allow us to focus on the global scale phenomenons: such as the interactions between neighboring regions, the time-scale of large events and the difference in the physical variables affecting the chlorophyll concentration in each region. In the following tasks, these simple predictive models will also be a reference for evaluating more complex ones. 

\subsection{Open Questions}

\begin{itemize}
\item Is the biological aggregation of the Red Sea proposed by (Raitsos 2013) statistically meaningful?
\item Can clustering methods be used to identify marine ecological zones based on chlorophyll data?
\item Can a simple forecasting model allow us to understand the causes of chlorophyll blooms?
\item Can the current hypothesizes about the seasonal chlorophyll dynamics be validated?
\end{itemize}

\subsection{Method}

Define datasets (training and test datasets, cross-validation).....................................0%
Variable selection (Lasso, L1 regression, single-variable linear regression)...............0%
Define regional aggregations (unsupervised learning, Hierarchical clustering, K-means)...................................................................................................................50%
Forecasts chlorophyll concentration (linear regression, GAM models, diagnostic, k-nearest neighbors)....................................................................................................0%
Predicting future extreme blooms (nearest-neighbours, logistic regression, decision trees)...........................................................................................................................0%

\subsection{Expected Outcomes}

A regional division of the Red Sea that has been quantitatively evaluated.
A critic of current hypothesis about the chlorophyll dynamics in the Red Sea.
A lower bound on the performance of a more sophisticated model.
An assessment of the limitation of aggregate methods for Chlorophyll data.
An understanding on how the treatment of spatial correlations can improve the results.

\section{Task 3: Global geostatistical model for forecasting}

Duration: 1 month (by March 2015)

Submission: Spatial Statistics

\subsection{Motivation}

Geostatistical methods can be used to construct dynamical models for forecasting the chlorophyll concentrations that we can compare to deterministic models. Geostatistics is a robust method to model spatio-temporal data. Recently there has been a lot of research on expanding it to model spatio-temporal data. With Kriging, these models a are powerful ways to do spatio-temporal prediction. As a particular case of Kriging, by predicting the spatial future field given the observation of the present field, we can derive a linear dynamical model. This linear model can be employed in a filtering setting like the Kalman filter. This is a desirable setup, as it is similar to the way deterministic models are employed to do forecasts given past observations. 

\subsection{Open Questions}

\begin{itemize}
\item Can a global geostatistical model fit chlorophyll data?
\item How non stationary is the data in time and space?
\item What spatiotemporal covariance functions best fit the chlorophyll data?
\item Can geostatistical methods be employed in a filtering setup?
\end{itemize}

\subsection{Method}

This task has already been started and had been the object of a submission for publication. The remaining work includes:
Use the new dataset and the new covariates
Compare the results to the ones of with the regional aggregates
Expected Outcomes
A methodology to employ a geostatistical model in a filtering problem.
A characterization of the space-time non stationarity of the data, and the interaction of the temporal and spatial dimensions.
An understanding of how spatial aggregation and geostatistical models can be used in the same model. 


\section{Task 4: Local geostatistical model for forecasting}

Duration: 3 months (by June 2015)
Submission: Journal of the American Statistical Association (Case Study)
Collaborator: Raphael Huser

\subsection{Motivation}

This part will bring together the results of the two preceding tasks to develop a predictive model that takes into account the large-scale dynamics and the regional spatio-temporal dynamics. In task 2, a predictive model is built, that represents the large scales behaviour of the Red Sea, but the spatial dimension inside each region is not addressed. We expect local features to play a role, such as the proximity to the coast, the bathymetry, proximity to other regions or major cities, etc. In task 3, we developed a methodology to use a geostatistical model in a dynamic fashion to do pixel-scale forecast. In this task, each regions will be modeled separately by a local geostatistical model that can do local prediction. These models will have access to aggregate covariates from neighboring regions in represent the global scale behaviours. 

\subsection{Open Questions}

What are the most adapted space-time covariance models for chlorophyll data?
How to use global covariates in a geostatistical model?
What are the differences in the fine-scale dynamics of chlorophyll in each region?
Can the fine scale behaviour of phytoplankton be predicted accurately?
What are the spatial features that are important for the chlorophyll dynamics?

\subsection{Method}

Extract local dataset from previous tasks
Design the training and test datasets, and the cross-validation method
Design and evaluate the mean function given the past covariates
Fit the local geostatistical model to the residuals.
Evaluate the model predictions and compare the results with task 2 and 3.
Expected Outcomes
A methodology to aggregate local geostatistical models
An improvements in the prediction skills over the models of task 2 and 3.
An understanding of the differences between each regions.
A critical evaluation of the space-time covariance models for fitting chlorophyll data.
A better characterization of the regional chlorophyll dynamics.


\section{Task 5: Assimilation of 1D ecological models and comparison to statistical models}

Duration: 3 months (by September 2015)

Submission: Journal of Geophysical Research 

Collaborator: George Triantafyllou / Boujeema

\subsection{Motivation}

The three previous tasks focus on constructing increasingly sophisticated predictive models for the chlorophyll concentration in the Red Sea. In this part these models will be compared to a 1D ecological model (ERSEM). This model is well detailed and very complex. The goal of this part will be to identify the merits of each modeling approach, and propose ways in which they can complement each other. To allow for comparison, the model will be run in each of the regions found in task 2. Available data will also be assimilated to the model through a smoothing assimilation scheme that will use an expectation-maximization algorithm for parameters estimation. 

\subsection{Open Questions}

Are statistical methods competitive for forecasting chlorophyll concentrations?
How can statistical and deterministic models complements each other?
Can statistical method forecast interesting dynamical features?
Are there significant regional differences in the relative performances of both approaches?
How to estimate the parameters of ecological models? 

\subsection{Method}

Define the metrics for comparison
Calibrate the ERSEM model on each of the regions
Define an assimilation scheme and the data for the ERSEM model
Implement the assimilation scheme
Run the simulation and aggregate the results
Do the comparisons with the statistical models

\subsection{Expected Outcomes}

A complete set of measures of the prediction skills of each approach.
A method to estimate assimilation and model parameters in an assimilated ecological model.
A set of case studies of the behaviours of each method for forecasting interesting events.
An understanding of the limitations of geostatistical models to predict nonlinear dynamics. 
Propositions on how the two approaches can complement each other.

\section{Task 6: Improving an Ecological Model Data Assimilation Scheme through Statistical Predictive Models}

Duration: 3 months (by September 2015)

Submission: Journal of Geophysical Research 

Collaborator: George Triantafyllou / Boujeema

\section{Motivation}

In the previous task, we compared the forecasts of the ecological ERSEM model to that of the statistical models we developed from tasks 2 to 4. In this task we will study how these two approaches can be complementary. Specifically, we will study the use of statistical forecasts model to improve the forecasts of the ERSEM ecological model. The forecasts of the statistical models will be treated as observations, that can be assimilated by the filtering scheme used with the ERSEM model, and will give an improved forecast. When real observations will be available, they will be assimilated sequentially. This, method will allow the different ERSEM models on each cluster to communicate indirectly their states to one another. 

\subsection{Open Questions}

Can statistical predictive models be used to communicate information between deterministic model?
Would the access to information about other regions improve the model forecasts?
What are the global patterns of ecological dynamics in the Red Sea?

\subsection{Method}

Define new assimilation scheme
Define metrics to measure model improvement
Prepare training and test datasets
Train statistical model
Run simulation with assimilation of statistical observation
Compare with results of task 5

\subsection{Expected Outcomes}

An improvement in the prediction skills of the deterministic approach
A methodology to couple deterministic ecological models through statistical models
Insights on the global ecological dynamics of the Red Sea


