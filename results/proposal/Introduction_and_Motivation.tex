\chapter{Introduction and Motivation}

\section{Phytoplankton and the Red Sea Ecology: Significance, Large-Scale
Features, and Applications}

\subsection{The Importance of Phytoplankton}

Phytoplankton are unicellular, free-floating, photosynthetic algae that live in
the upper layers of bodies of water (ocean, lakes, rivers or ponds). There
exists a wide diversity of phytoplankton species. Up to date, about 5000 of
them have been identified \citep{Tett1995}. Phytoplankton are also highly
variable in sizes, ranging from \SI{0.2}{\micro\metre} for cyanobacteria to
\SI{200}{\micro\metre} for the largest species of diatom \citep{Pal2014}. In
the oceans, phytoplankton live in the surface layer where there is enough
sunlight for photosynthesis. 

Phytoplankton play a fundamental role for the ocean ecology. They are at the
basis of the marine food web and trap most of the energy used by pelagic
ecosystems \citep{Pal2014}. Zooplankton graze phytoplankton, which are in turn
consumed by higher trophic levels. It has been estimated that nearly 98\% of
the ocean primary productivity comes from phytoplankton \citep{Pal2014}.
Phytoplankton are also responsible for maintaining the dissolved oxygen level
necessary for other species to survive \citep{Pal2014}. High phytoplankton
concentration however may also negatively impact their environment by creating
dead zones. When they die and sink, the bacteria that decompose them can
consume all the available oxygen \citep{Pal2014}, and this may cause massive
mortality in the fauna. Moreover, because of its short life cycle,
phytoplankton respond very well to changes in its environment, making it a key
parameter to monitor water quality \citep{Wu2014}.

Phytoplankton place at the bottom of the marine food chain makes it an
important factor for fisheries. Highly productive fishing zones such as the
regions in the Arabian seas, Californian coast, north-west African coast and
Chilean coast, are related to the upwelling of cold nutrient rich water
favourable to phytoplankton growth \citep{Mann2006}. As such, remotely-sensed
chlorophyll data have been routinely used since the last decade to help
fisheries predict the timing of phytoplankton blooms \citep{Robinson2010}. The
El-Nino phenomenon is known to create less favourable conditions for
phytoplankton in the Eastern Pacific, resulting in a dramatic reduction of fish
catches in the western coast of South America
\citep{Robinson2010}. In contrast, in the Red Sea, the MEI (Multivariate ENSO
Index) has been found to positively correlate with chlorophyll concentration,
and this may have a important implications for regional fisheries
\citep{Raitsos2015}.

Phytoplankton also plays the role of a biological CO2 pump and strongly impact
the Earth climate \citep{Mann2006}. During photosynthesis, phytoplankton
captures carbon and releases oxygen. A part of this organic material stays in
the food web, either transmitted to higher trophic level, or degraded  by
bacteria. Another part sinks to the bottom of the ocean to form sediments. It
is estimated that phytoplankton accounts for 48\% of the Earth carbon fixation
\citep{Pal2014}.

\subsection{Red Sea Large-Scale Phytoplankton Dynamics}

Typical tropical seas (TTS), such as the Red Sea, are characterized by a highly
stratified structure, where warm nutrient-depleted surface water is separated
from the cold nutrient-rich deep water by a steep gradient of temperature zone
called pycnocline. The pycnocline acts as barrier that limits the upward
nutrients flow \citep{Mann2006}. As a result, TTS are oligotrophic, exhibiting
low chlorophyll concentrations. Until recently, marine biologists believed that
tropical and subtropical seas have low productivity. However, recent
investigations have contested this idea, suggesting that different upwelling
mechanisms (winter deep mixing, storms, eddies, etc) occur in these regions,
that bring new nutrients to the surface water \citep{Mann2006}.

Despite being an oligotrophic and challenging environment for marine life, the
Red Sea has a surprisingly rich and diverse ecosystem \citep{Raitsos2011}, and
a very well developed coral reef system \citep{Racault}. The source of nutrient
for sustaining such a developed ecosystem is not well understood yet, but the
exchange with the open ocean, atmospheric depositions and transport through
the large number of mesoscale eddies in its basin are believed to be major
factors \citep{Raitsos2013, Zhan2014}.

Although the Red Sea environment is still relatively well preserved, it is
increasingly stressed by human activities. The continuous urbanization and
fishing activity contribute to the fragilization of this unique ecosystem
\citep{Acker2008}. An abrupt increase of temperature has further occurred in
the last decade, which may further threaten its fragile coral reef system
\citep{Raitsos2011}.

Because of the lack of in-situ measurements, the large-scale phytoplankton
dynamics of the Red Sea remain largely unknown \citep{Raitsos2013,
Triantafyllou2014}. In very recent studies, remotely-sensed data and computer
simulations have been used to improve our knowledge of the ecology of this
region \citep{Raitsos2013, Racault, Triantafyllou2013}. The Red Sea is
deficient in major nutrients \citep{Weikert1987}, and the only significant
input of nutrient-rich water comes from the Gulf of Aden \citep{Yao2015}. This
explains the general pattern of chlorophyll concentration increase from south
to north \citep{Raitsos2013}, as can be seen in Figure \ref{meanchl}.  The Red Sea
also displays a distinct seasonality, with a peak in concentration during
winter.  A weak summer peak is also observed over the whole Red Sea around
July, except in the northernmost region \citep{Raitsos2013}. Despite this
regularity, a strong interannual variability was observed, with blooms that may
reach mesotrophic concentration levels \citep{Raitsos2013}. According to
\citet{Triantafyllou2014}, the variations in the Red Sea ecology are mainly
driven by physical circulation. In the rest of this section, we explore some of
the mechanisms that have been associated to the major features of chlorophyll
concentration in the Red Sea.

\begin{figure}[h]
    \centering
    \includegraphics[scale=.15]{figures/chl_average.png}
    \caption{Average chlorophyll concentration from CCI data}
    \label{meanchl}
\end{figure}

The exchange of water with the nutrient-rich Gulf of Aden is a major driving
mechanism of the Red Sea productivity \citep{Triantafyllou2014}. It is believed
to be the most important source of nutrients. The maximum chlorophyll
concentration observed in the southern Red Sea during winter is attributed to
wind-driven water intrusion \citep{Raitsos2013, Yao2014}. In Summer, this
exchange of water is believed to be the major source of nutrients for the whole
Red Sea. The influence of the water intrusion weakens as the latitude
increases, explaining the low concentration in the northern half of the Red Sea
\citep{Raitsos2013}.

Deep convection also plays an important role in allowing nutrient-rich deep
water to mix with water of the euphotic zone. Vertical mixing is the most
vigorous in the northern extremity of the Red Sea during winter. This
explains its higher chlorophyll concentration compared to the north-central Red
Sea, a region of weak mixing \citep{Raitsos2013}. The northern Red Sea mixing
is believed to be mainly driven by wind \citep{Raitsos2013}.

The Red Sea circulation is strongly influenced by mesoscale eddies
\citep{Yao2014, Yao2014b, Zhan2014} which may have important impact on its
primary productivity \citep{Zhai2013}. In particular, the anti-cyclonic eddy in
the central Red Sea is believed to drive the June concentration peak and the
summer productivity of this region, by transporting nutrients and/or
phytoplankton from the adjacent coral reefs \citep{Raitsos2013}. In the
northern Red Sea, a cold-core eddy plays a role in enhancing the vertical
mixing in this region \citep{Raitsos2013}.

Aerial depositions of dust could also be an important input of nutrients into
the Red Sea, but it has been largely left unexplored so far
\citep{Triantafyllou2014}.
\citet{Raitsos2013} noticed for example that sand storms in the Red Sea most
frequently happen in June and July, which coincides with the summer chlorophyll
peak. Finally, climate mode indices have been shown to be strongly correlated
with air-sea heat exchanges in the Red Sea \citep{Abualnaja2015}, and might
therefore influence its biology. This has also been recently confirmed in
another study by \citet{Raitsos2015}, who have shown that El Nino has a
positive impact on the chlorophyll concentration, by strenghtening the wind
transporting nutrients into the Red Sea from the Gulf of Aden.

\section{Remotely-Sensed Chlorophyll Data: Relevance and Challenges for the Red
Sea}

\subsection{Measuring Chlorophyll Concentration}

Chlorophyll is a molecule present in algae, phytoplankton and plants that is
critical for photosynthesis. It is a poor absorber of green light, and is
responsible for the coloration of plants \citep{Pal2014}. When phytoplankton
are present in high concentrations, the water usually takes a detectable green
coloration (it may also take a red or blue coloration depending on the type of
dominating phytoplankton) \citep{Robinson2010}. This offers an efficient way to
measure the phytoplankton concentration from space \citep{Robinson2010}.

In-situ measurement of chlorophyll concentration can be gathered through
scientific cruises, buoy stations or gliders (unmanned submarines). These
methods are expensive to deploy and therefore generally have limited temporal
and spatial coverage \citep{Robinson2010}. Governmental restrictions,
as in the Red Sea,
as well as security issues, as in the Arabian Sea, set also barriers to in-situ
measurements.

Satellite measurements of chlorophyll provide excellent proxies for
phytoplankton concentrations with a good temporal and spatial coverage
\citep{Robinson2010}. The SeaWIFS, MODIS and MERIS missions have provided an
uninterrupted coverage of global ocean since 1997. High-resolution maps of
daily chlorophyll concentration are freely accessible to the scientific
community \citep{McClain2009}. Despite some limitations, such as missing data
due to cloud coverage and sunglint, or problematic values in coastal areas,
remotely-sensed chlorophyll concentrations are intensively used by the
scientific community. In regions, like in the Red Sea, where little in-situ
measurements are available, these constitute the most important data source
\citep{Raitsos2013, Brewin2013}.

\subsection{Limitation of Remotely-Sensed Chlorophyll Data in the Red Sea}

The quality of remotely-sensed chlorophyll data products such as MODIS and
SeaWiFS in the Red Sea is comparable to that of the rest of the global ocean for
case I waters (open sea) \citep{Brewin2013}. However, the data contains a large
amount of missing values because of persistent clouds, sun-glint and sensor
saturation \citep{Racault}. This problem is particularly acute during the
summer in the southern Red Sea where the data coverage is almost null
\citep{Racault}, as can be seen from Figure \ref{misval_modis}.

Chlorophyll concentration estimation in optically complex case II waters is a
recurrent problem in this remotely-sensed data that particularly affects the
southern Red Sea.  In this region, the remotely sensed chlorophyll data could
be overestimated \citep{Raitsos2013}. However, all high values are not
necessarily bad, as highly productive coral reefs can be expected in this
region \citep{Raitsos2013}. These values have however not been validated yet,
mainly because of the lack of in situ measurements \citep{Raitsos2013}.

\begin{figure}[h]
    \centering
    \includegraphics[scale=.15]{figures/modis_missing_values_summer.png}
    \caption{Percentage of missing values in the MODIS chlorophyll dataset}
    \label{misval_modis}
\end{figure}

One solution to missing and bad values is to apply a data filling algorithm, of
which one of the most popular algorithms is the Date INterpolating Empirical
Orthogonal Functions (DINEOF), which is an EOF based data filling approach
introduced by \citet{Beckers2003}. In \citet{Sicarjobs2011}, it has been
employed to fill chlorophyll data with 70\% of missing values.
\citet{Taylor2013} compared DINEOF with other EOF-based reconstruction
algorithms, suggesting that the former is a better method for data filling.
DINEOF has been further employed in several other chlorophyll studies
\citep{Miles2010, Waite2013}, and has also been used for multivariate
reconstruction of SST fields using chlorophyll data in \citet{Alvera2007}. 

The Ocean Colour, Climate Change Initiative (OC-CCI) is a new chlorophyll
data product\footnote{http://www.esa-oceancolour-cci.org} that considerably
increases the Red Sea coverage. It is a merged product from the SeaWiFS, MODIS
and MERIS data missions.  Overall, it achieves a 75-80\% coverage in the entire
Red Sea basin against 50-65\% for a single sensor \citep{Racault}. During the
summer, the improvement in coverage is dramatic, as shown in Figure
\ref{misval_cci}.  This is mostly due to the use of the POLYMER algorithm that
allows to exploit the MERIS data collected during hazy conditions
\citep{Steinmetz2011}. This new dataset has not been fully investigated to
revisit the assumptions made on the large-scale Red Sea phytoplankton
productivity.

\begin{figure}[h]
    \centering
    \includegraphics[scale=.15]{figures/cci_missing_values_summer.png}
    \caption{Percentage of missing values in the CCI chlorophyll dataset}
    \label{misval_cci}
\end{figure}

\section{Modeling and Forecasting Chlorophyll: Data and Dynamics Driven
Approaches, and Applications}

\subsection{Why Modeling Chlorophyll?}

Computer models are useful to identify causes behind observed chlorophyll
patterns.  Many hypotheses have been made about the drivers of chlorophyll
concentration in this regions, but some of them have not been investigated
through models until very recently. The role played by the exchange of water
with the Gulf of Aden and winter mixing in the northern Red Sea have been
successfully modeled with a 3D coupled physical-ecological model
\citep{Yao2014, Yao2014b, Triantafyllou2014}.  However, the interaction between
the open sea and coral reefs, and the role of atmospheric depositions have not
been investigated yet.  Models, can also be helpful for understanding governing
dynamics affecting the chlorophyll concentration. In particular, the
interaction between the productivity level of the different regions of the Red
Sea is yet to be explored.

Model predictions of chlorophyll concentration also have practical applications
for fisheries. Furthermore, phytoplankton blooms can be harmful to humans and
marine life and are closely monitored in many regions of the world
\citep{Pettersson2013}. In the Red Sea, where tourism and aquaculture are
developing, it is likely to become a concern too. Phytoplankton is also
directly, and indirectly through zooplankton, the cause of microfouling that
affects desalination plants. In 2008-2009, a red tide forced the shutdown of
desalination plants along the Gulf of Oman and the Arabian Gulf
\citep{Richlen2010}.


\subsection{Deterministic Models}

\subsubsection{Ecological Models}

There is a rich literature on the modeling of marine ecosystems based on a set
of mathematical equations governing the ecosystem variability and dynamics (see
\citet{Fennel2004} for an introduction). In these models the interactions of
complex physical, chemical and biological processes are modeled by differential
equations that represent the flow of carbon, nitrogen, phosphate and silicon.
The biota is usually divided into trophic levels, and can be
further divided by feeding methods and size classes \citep{Baretta1995,
Triantafyllou2014}.

Ecological deterministic models vary widely in complexity depending on the
retained number of chemical, biological and ecological processes. These can
be as simple as the nutrient-phytoplankton-zooplankton (NPZ) model
\citep{Anderson2005} that only uses three variables representing two trophic
levels and nitrate, or as complex as the European regional seas ecosystem model
(ERSEM) that included dozens of variables \citep{Baretta1995}. NPZ models are
extensively used because of their simplicity and capacity to model the
large-scale features of marine ecosystems \citep{Anderson2005}. ERSEM has
recently been coupled to the MITgcm circulation model to simulate the Red Sea
ecology \citep{Triantafyllou2014}.  The complexity of these models makes them
difficult to parametrize when not enough data are available, which is usually
the case in many marginal seas \citep{Anderson2005}. Ecological models can be
subject to important sources of uncertainties. These can be due to imperfect
modeling assumptions, poorly known initial conditions and physical forcing, and
approximative parametrization schemes \citep{Edwards2015}.  These set important
limitations on the ecosystem models simulations, often causing large deviations
of the models outputs form reality.

\subsubsection{Data Assimilation}

Data assimilation is used to improve the simulations of ecological dynamical
models and enhance their forecasting capabilities by constraining their
predictions with available observations \citep{Edwards2015}. Such prediction
capabilities are now deployed in operational systems, for example to assess
the impact of human activities on the ecosystem of the Gulf of Pagasitikos
\citep{Korres2012}. The deployment of a similar forecasting system in the Red
Sea is currently under development \citep{Triantafyllou2014}. Hindcasting, the
estimation of unobserved variables, is another application of assimilation
systems. \citet{Ciavatta2011} showed that data assimilation of ocean color data
may improve the seasonal and annual hindcast of non-assimilated biogeochemical
properties in the shelf area of Western English Channel. Data assimilation can
also be used for reanalysis, in order to provide estimates of biogeochemical
distributions of the past \citep{Fontana2013}.

In the marine ecology modeling community, three assimilation schemes have been
widely used: the Ensemble Kalman filters (EnKF), the Singular Evolutive
Extended Kalman filter (SEEK), and its ensemble variant, the Singular Evolutive
Interpolated Kalman filter (SEIK). The EnKF, a Monte Carlo-based approximation
of the Kalman Filter, has been used in \citet{Ciavatta2011, Ciavatta2014}. This
scheme may however suffer from sampling errors when the ensemble size is
smaller than the number of observations, as is usually the case when
assimilating remotely-sensed data \citep{Nerger2005, Altaf2014}.  SEEK is a
reduced-rank variant of the Extended Kalman filter (EK). It was introduced for
efficient data assimilation into large scale ocean models.  It is based  on the
projection of the error covariance onto a low dimensional space.  SEEK has a
long history in data assimilation into marine ecology models and is still
extensively used in recent studies \citep{Fontana2013, Korres2012,
Butenschon2012}. SEIK is an ensemble variant of the SEEK  and a deterministic
version of the EnKF that do not suffer from observations sampling errors, as it
updates the filter forecast and its covariance exactly as in the Kalman filter.
It therefore requires a resampling step to generate a new analysis ensemble for
the next forecast step. SEIK has been used by \citet{Triantafyllou2013,
Korres2012}. \citet{Korres2012} shows that SEIK and SEEK are both comparably
robust assimilation methods for highly nonlinear systems. \citet{Hoteit2005}
has suggested however that SEIK outperforms SEEK using a highly nonlinear
model.

Predicting the state of ecological models is a challenging problem for
state-of-the-art data assimilation schemes \citep{Edwards2015}. First,
biogeochemical variables are usually positive concentration, whereas Kalman
filters expect Gaussian variables, and log-transformation may fail to avoid
this issue \citep{Ciavatta2011}. In an attempt to mitigate this problem,
\citet{Fontana2013} introduced Gaussian anamorphosis transformations.  Second,
ecological blooms are intermittent and highly nonlinear, conditions that are
challenging for Kalman filter-based assimilation schemes \citep{Hoteit2005}.
Third, SEIK, EnKF and SEEK project the error covariance onto some subspace,
often resulting in an underestimation of the estimation error. \citet{Butenschon2012}
studied different ways to propagate the error covariance in order to alleviate
this issue. Finally, the model error statistics are required by Kalman-derived
filters, but are difficult to estimate. \citet{Triantafyllou2013} proposed to
use the $H_\infty$ method with SEIK in order alleviate this requirement.

Particle filters represent a class of data assimilation schemes that, unlike
Kalman-based filters, do not require any linearity or Gaussianity assumption
\citep{Edwards2015}. As such, they might be more suitable for data assimilation
into ecosystem models. Particle filters have been studied in the case of 0D and 1D
ecological models \citep{Edwards2015}, but are still strongly limited by their
demanding computational requirements. The possibility of applying Particle
filters to large-slace models is currently an active field of research in the
data assimilation community \citep{Edwards2015}.


\subsection{Data-Driven Approaches}

Compared to data assimilative ecological models, data-driven statistical models
are relatively simpler to develop. Such models are usually
relevant when the phenomenon
producing the data is dynamically complex to model, or simply poorly understood
\citep{Gareth2013}. Various data-driven models have been proposed
to predict chlorophyll
concentration, mostly in small regions with complex dynamics (see references
below). Some statistical models, such as linear regression, Gaussian additive
models, or tree regression have the advantage of being easy to interpret
\citep{Gareth2013}, and could be used to understand the dynamics driving the
chlorophyll concentration \citep{Raitsos2012}.

\subsubsection{Statistical Models}

Statistical and machine learning models have been recently used for estimation and
classification problems related to phytoplankton concentrations. One
application is the detection of harmful algal bloom from spatio-temporal
satellite dataset, which has been addressed by \citet{Gokaraju2011} in the Gulf
of Mexico using support vector machines. Another application is the estimation
of chlorophyll concentration in case II coastal water using satellite radiance
data. This problem was considered by \citet{Kim2014} in the west coast of
South Korea, and by \citet{Camps-Valls2006} using a global dataset of in situ
measurements.  The former applied the support vector regression algorithm, while
the latter used the random forest algorithm.

Machine learning algorithms, in particular Artificial Neural Networks,
are also very popular for forecasting regional chlorophyll concentration in regions
with very complex dynamics. Neural networks are also commonly used for forecasting
chlorophyll concentration in fresh as well as in coastal water systems. In
\citet{Jeong2006}, temporal recurrent recursive neural networks have been used
and found superior to traditional time-series models for daily forecasts of
chlorophyll concentration.  \citet{Wang2013} also used recurrent neural
networks for forecasting daily chlorophyll in Lake Taihu, China.
\citet{Mulia2013} combined Neural Network and genetic algorithm for nowcasting
and forecasting of the chlorophyll concentration up to two weeks ahead, in the
tidal dominated coast of Singapore.  Finally, \citet{Lee2003} used neural
networks for the forecasting of algal bloom with one or two weeks lags in the
coastal waters of Hong-Kong.

\subsubsection{Geostatistics}

Phenomena such as propagation and diffusion of nutrients and phytoplankton
play a key role in the chlorophyll
spatial concentration, but are difficult to represent without spatial modeling.
There is also a difference in the chlorophyll patterns of different regions of
the Red Sea, in particular between the nutrient rich southern Red Sea and the
oligotrophic northern Red Sea, and between the open ocean and the coastal
waters \citep{Raitsos2013}.  One should also expect the different regions of
the Red Sea to interact horizontally.

In contrast with the statistical models discussed above, in geostatistical
models, the spatial dimension is modeled explicitely by representing the data
as a spatial stochastic process \citep{Gneiting2007}. In classical
geostatistics, spatial data is modeled as the realization of a two- (or three-)
dimensional Gaussian process \citep{Gneiting2007}.  Geostatistics can be easily
extended to spatio-temporal datasets \citep{Gneiting2007}. Many ways for
constructing space-time covariance functions for these models have been
recently proposed \citep{Gneiting2002, Cressie1999, Stein2005}.

The theory of space-time geostatistics is closely related to that of spatial
statistics, where the time dimension is treated as an additional dimension.
However, the relationship between these two dimensions is derived from a
dynamical process, that must be taken into account in the definition of the
covariance function \citep{Gneiting2010}. Some space-time covariance models can
actually be derived from a physical formulation, such as the frozen fields
\citep{Gneiting2010}, or stochastic differential equations \citep{Brown2000,
North2011}.

Physically-derived space-time covariance functions are not commonly used, and
the usual approach is to construct them from spatial and temporal covariance
functions \citep{Gneiting2010}. One of the simplest forms of covariance
function are the separable covariance functions, that are the products of a
spatial covariance function and temporal covariance function. These are
computationally efficient, but are not suitable to represent space-time
interactions \citep{Cressie1999, Stein2005}, making them of limited use for
modeling physical systems. The Cressie, Huang spectral characterization theorem
of space-time covariance functions has opened the door to wider ways of
constructing them. For example, \citet{Gneiting2002} presented a simple
criterion that allows their construction from a very large class of models. 

Space-time geostatistical models have been used in a variety of applications.
\citet{Hohn1993} used it for forecasting the outbreaks of an invasive specie.
These methods have also been used in meteorology to model temperature fields
\citep{Handcock1994, North2011} or wind \citep{Cressie1999, Gneiting2002}, and
in environmental studies for ground-level ozone concentration.  To the best
of our knowledge, these methods
have not been investigated for modeling and forecasting chlorophyll
concentration.

\section{Thesis Objectives}

The Red Sea ecological processes are complex and poorly
understood. Even though ocean color remote-sensing data has revolutionized our
understanding of marine ecology, it only provides information about the
phytoplankton at the sea surface. Our knowledge is even more limited in the Red
Sea, due to the paucity of in situ data. This makes the modeling and
forecasting of chlorophyll variability in the Red Sea a very challenging
problem.

The goal of this thesis is to develop novel models for efficient forecasting of
chlorophyll concentration in the Red Sea. Various statistical and deterministic
models will be developed and
tested following data and a dynamical-driven approaches. In particular, we will
develop efficient approaches to model the Red Sea chlorophyll,
introduce the use of geostatistics to the field of marine
ecology, and explore the
possibility of improving the state-of-the-art data assimilative ecosystem
modeling. The merits of both approaches will be compared for the first time in
the same region, and under the sames conditions. We will also 
investigate and propose a new framework to
efficiently combine both approaches methods for best forecasting of chlorophyll
concentration.

Both deterministic and statistical approaches have been used to
predict chlorophyll. However, these approaches have never been compared, nor
combined in the same problem. A thorough comparison is very much needed. It
would provide modelers with a comprehensive foundation for developing their
modeling strategy. Combining them is also expected to further improve the
forecasting skills, and this will be thoroughly investigated.

This thesis will also help improving our understanding of the Red Sea ecology.
In particular, it will identify possible drivers of the chlorophyll
seasonality and interannual variability. We will also identify and characterize
the different Red Sea eco-regions. This work has practical applications for the
region. In particular, better forecasting phytoplankton blooms may be extremely
important for the fisheries and the operations of the increasing aquaculture activies and desalination plants along the coast of the Red Sea.
