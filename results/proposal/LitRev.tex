\chapter{Literature Review}

\section{Red Sea large-scale phytoplankton dynamics}

Because of the lack of in-situ data, the large-scale phytoplankton dynamics of the Red Sea remain largely unknown \cite{Raitsos2013, Triantafyllou2014}. However, in recent studies, remotely-sensed data and computer simulations have been used to improve our knowledge of the biology of this region. The Red Sea is deficient in the major nutrients \cite{Weikert1987}, and the only significant input of water comes from the Gulf of Aden. This explains a general increase of chlorophyll concentration from north to south \cite{Raitsos2013}. The lowest concentration is found in the northern central Red Sea. The Red Sea also displays a distinct seasonality, with a peak in concentration during the winter. A weak summer peak is also observed around July, everywhere except in the northernmost region \cite{Raitsos2013}. Despite this regularity, a strong interannual variability is observed, with blooms that can reach mesotrophic concentration levels \cite{Raitsos2013}. According to \cite{Triantafyllou2014}, the variations in the Red Sea ecology are mainly driven by circulation. In the rest of this section, we explore some of the mechanisms that have been linked to the major features of chlorophyll concentration.


The exchange of water with the nutrient-rich Gulf of Aden is a major driving mechanism for the whole Red Sea \cite{Triantafyllou2014}. It is the most important source of nutrient. The maximum chlorophyll concentration observed in the southern Red Sea during winter is attributed to wind-driven water intrusion \cite{Raitsos2013}. In Summer, this exchange of water is believed to be the only significant source of nutrients for the whole Red Sea. The influence of the water intrusion weakens as the latitude increases, explaining the low concentration in the northern half of the Red Sea \cite{Raitsos2013}.

Deep convection also plays an important role in allowing nutrient-rich deep water to mix with water of the euphotic zone. The vertical mixing is the most vigorous in the northern extremity of the Red Sea during the winter. This explains its higher chlorophyll concentration compared to the north-central Red Sea, a region of weak mixing \cite{Raitsos2013}. The northern Red Sea mixing is believed to be driven by wind \cite{Raitsos2013}.

The Red Sea circulation is strongly influenced by mesoscale eddies \cite{Yao2014, Zhan2014} that could impact primary production \cite{Zhai2013}. In particular, the anti-cyclonic eddy in the central Red Sea is believed to control the June concentration peak and the summer productivity of this region, by transporting nutrients and/or phytoplankton from the adjacent coral reefs \cite{Raitsos2013}. In the northern Red Sea, a cold-core eddy plays a role in enhancing the vertical mixing in that region \cite{Raitsos2013}.

Aerial depositions of dust could also be an important input of nutrient for the Red Sea, but it has been largely left unexplored \cite{Triantafyllou2014}. \cite{Raitsos2013} noticed for example that sand storms in the Red Sea most frequently happen in June and July, which coincides with the summer chlorophyll peak. Finally, climate mode indices have been shown to be strongly correlated with air-sea heat exchanges in the Red Sea \cite{Abulnaja2015}, and might therefore influence its biology.

\section{Challenges with remotely-sensed chlorophyll data}

Chlorophyll remotely-sensed data in the Red Sea suffer from many problems that need to taken into account before using them. Until recently, the data coverage in the southern Red Sea was almost 0\% during the summer due to sunglint, clouds and aerosols \cite{Racault}. However the OC-CCI data product that merges different chlorophyll data sources considerably increases the Red Sea coverage. However, this new dataset has not been used to revisit the assumption made in the large-scale Red Sea phytoplankton productivity. If for case I open water, the data in the Red Sea has been shown to be of quality comparable with  that the rest of the world \cite{Brewin2013}, case II waters remain a challenge, especially in the SRS where the water is particularly shallow and the coral reef extended \cite{Triantafyllou2014, Raitsos2013}. Theses problems generally lead to an overestimation of the chlorophyll level, but not necessarily to erroneous values \cite{Raitsos2013}.

One of the most popular algorithm for the data filing of remotely-sensed data is DINEOF. It is an EOF based data filling approach introduced by \cite{Beckers2003}. It has been used for multivariate reconstruction of SST fields using chlorophyll data in \cite{Alvera2007}. In \cite{Sicarjobs2011}, it has been employed to fill chlorophyll data with 70\% of missing values. [Taylors 2013] has compared DINEOF with other EOF-based reconstruction algorithms and shown that the former is the best method for data filling. DINEOF has been employed in several other chlorophyll studies \cite{Miles2010, Waite2013}.

\section{Data Assimilation for marine ecology models}

Data assimilation schemes are used to improve the simulations of ecological dynamics models by correcting their predictions with observations. The most common use of data assimilation is to improve the forecast of an ecological model, by providing it with an estimated initial state. Such prediction capabilities are deployed in operational expert systems, for example to study the impact of human activities on the ecosystem of the Gulf of Pagasitikos \cite{Korres2012}. The deployment of such a forecasting system in the Red Sea is currently under study \cite{Triantafyllou2014}. Hindcasting, the estimation of unobserved variables, is another application of assimilation scheme. \cite{Ciavatta2011}  showed that they could improve the seasonal and annual hindcast of non assimilated biogeochemical properties in a shelf area (Western English Channel). Finally, data assimilation can be used for reanalysis, to provide estimates of past years biogeochemical variables \cite{Fontana2013}. 

In the marine ecology modeling community, two assimilation schemes have been widely used: the Ensemble Kalman filters (EnKF) and the Singular Evolutive Extended Kalman filter (SEEK). The Stochastic EnKF, a Monte-Carlo approximation of the Kalman Filter, has been used in \cite{Ciavatta2011, Ciavatta2014}. However, it suffers from sampling errors when the ensemble size is smaller than the number of observations, as is usually the case when assimilating remotely-sensed data. The Singular Evolutive Interpolated Kalman filter (SEIK) is a deterministic version of the EnKF that do not suffer from sampling problems, as it projects the propagated error in a low-dimensional subspace. SEIK has been used by \cite{Triantafyllou2012, Korres2012}. SEEK is a reduced order version of the Extended Kalman filter (EK), that in intractable in high-dimensions. Like SEIK, it projects the error covariance in a low dimensional space. SEEK has a long history in data assimilation for marine ecology models and is still used in recent studies \cite{Fontana2013, Korres2012, Butenschon2012}. \cite{Korres2012} shows that SEIK and SEEK are both comparably robust methods for highly non linear systems.

Current assimilation schemes are however affected by problems that have been addressed in the past years. First, biogeochemical variable are usually positive concentration, whereas Kalman filters expect Gaussian variables, and log-transformation can fail at solving this issue \cite{Ciavatta2011}. However, \cite{Fontana2013} has successfully introduced Gaussian anamorphosis transformations to solve this issue. Second, ecological blooms are intermittent and highly nonlinear, conditions that are challenging for assimilation schemes \cite{Triantafyllou2012, Korres2012}. Third, SEIK and SEEK both project the error covariance in a subspace, resulting in an underestimation of the estimation error. \cite{Butenschon2012} studied different ways to propagate the error covariance in order to alleviate this issue. Finally, the model error statistics are required by Kalman-derived filters, but are difficult to estimate. \cite{Triantafyllou2012} proposes to use the $H_\infty$ method with SEIK in order remove this requirement.

\section{Statistical models for chlorophyll concentration}
\label{LR_stats}

Statistical and machine learning models have been used for estimation and classification problems related to phytoplankton concentrations. One application is the detection of harmful algal bloom from spatio-temporal satellite dataset, that has been addressed in \cite{Gokaraju2011}, in the Gulf of Mexico, using support vector machines. Another application is the estimation of chlorophyll concentration in case II coastal water using satellite radiance data. This problem has been addressed by \cite{Kim2014} on the west coast of South Korea, and by \cite{Camps-Valls2006} using a global dataset of in situ measurements. The former used the support vector regression algorithm, while the latter used also the random forest algorithm.

Machine learning algorithms, in particular Artificial Neural Networks have been very popular for forecasting regional chlorophyll concentration in regions with very complex dynamics. In such regions, deterministic ecological models are usually too complicated to use and less efficient than data-driven approaches. Neural networks have been widely used for forecasting chlorophyll concentration in fresh as well as in coastal water systems. In \cite{Jeong2006}, temporal recurrent recursive neural network have been used and found superior to traditional time-series model for daily forecasts of chlorophyll concentration. \cite{Wang2013} also used recurrent neural networks for daily chlorophyll forecasting in Lake Taihu, China. \cite{Mulia2013} combined Neural Network and genetic algorithm for nowcasting and forecasting of the chlorophyll concentration up to 14 days ahead, in the tidal dominated coast of Singapore. Finally, \cite{Lee2013} used neural networks for the forecasting of algal bloom with one and two weeks lags in the coastal waters of Hong-Kong.

\section{Space-time geostatistical models for forecasting}

The theory of space-time geostatistics is closely related to that of spatial statistics. In fact, the time dimension is an additional dimension. However, the time and space interactions derive from physical interaction, and must be taken into account in the definition of the covariance function \cite{Gneiting2010}. Some space-time can actually be derived from a physical formulation, such as the frozen fields \cite{Gneiting2010}, or SDEs \cite{Brown2000, North2011}.

Despite their theoretical interest, physically-derived space-time covariance function have been little used \cite{Gneiting2010}. More popular, are covariance functions built from simple building blocks. One of the most simple types are separable covariance functions, that are the product of a spatial covariance function and temporal covariance function. They are computationally efficient, but are enable to represent space-time interactions \cite{Cressie1999, Stein2005}, making them of limited use for modeling physical systems. The Cressie, Huang spectral characterization theorem of space-time covariance functions has opened the door to wider ways of constructing them. For example, \cite{Gneiting2002} presented a simple criterion that allows their construction from a very large class of models. 

Space-time geostatistical models have been use in a variety of applications. \cite{Hohn1993} used it for forecast the outbreaks of an invasive specie. They have been used in meteorology to model temperature fields \cite{Handcock1994, North2011} or wind \cite{Cressie1999, Gneiting2002}, and in environmental studies for ground-level ozone concentration. \cite{Gneiting2007, Gneiting2010} present recent more details on the theory of space-time geostatistics and its applications.  

