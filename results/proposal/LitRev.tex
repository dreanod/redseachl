\chapter{Miscellaneous}
\label{chapter4}

\section{Red Sea large-scale phytoplankton dynamics}

Despite an increasing number of study for the last two decades, the large-scale phytoplankton dynamics of the Red Sea remains largely unknown [Raitsos 2013, Triant 2014]. The Red Sea is deficient in the major nutrients [Weikert 1987], and the only significant input of fresh water comes from the Gulf of Aden. This explains a general increase of chlorophyll concentration from north to south with the NCRS having the lowest concentration [raitsos 2013]. The Red Sea also displays a distinct seasonality, with the highest chlorophyll concentration in winter. A minor summer peak is also observed around July, everywhere except in the NRS [Raitsos]. Despite this regularity, a strong interannual variability is observed, with blooms that can reach mesotrophic concentration levels [raitsos 2013]. According to [Triant 2014], the variations in the Red Sea ecology are mainly driven by circulation. In the rest of this section, we explore some of the mechanisms that have been linked to the major features chlorophyll concentration.

The exchange of water with the nutrient-rich Gulf of Aden is a major driving mechanism for the whole Red Sea [Triant. 2014]. It is the most important source of fresh water and nutrients. The maximum chlorophyll concentration observed in the SRS during the summer is due to the wind-driven water intrusion [Raitsos 2013]. In Summer, this exchange of water is believed to be the only significant source of nutrients for the whole Red Sea. The influence of the water intrusion weakens in as the latitude increases, explaining the lowest concentration in the northern half of the Red Sea [Raitsos 2013].

Stratification and deep convection also play an important role in allowing nutrient-rich deep waters to mix with waters of the euphotic zone. The vertical mixing is the most vigorous in the NRS during the winter, explaining why its chlorophyll concentration is higher than in the NCRS, where there is a lack of mixing [Raitsos 2013]. The NRS mixing is believed to be driven by wind [Raitsos 2013].

The Red Sea circulation is strongly influenced by mesoscale eddies that impact primary production [KIM 2011].  In particular, the anti-cycloninc eddy in the CRS is believed to control the June concentration peak and summer productivity in this region, by transporting nutrients and/or phytoplankton from the adjacent coral reefs [Raitsos 2013]. The cold core eddy in the NRS also plays a role in enhancing the vertical mixing in that region.

Aerial depositions could also be an important input of nutrient for the Red Sea, but it has been largely left unexplored [Triant 2014]. [Raitsos 2013] noticed for example that the Sand storms in the Red Sea most frequently happen in June and July, coinciding with the summer chlorophyll peak. 

\section{Challenges with remotely-sensed chlorophyll data}

Chlorophyll remotely-sensed data in the Red Sea suffer from many problems that need to taken into account before using them. Until recently, the data coverage in the southern Red Sea was almost 0\% during the summer due to sunglint, clouds and aerosols [Racault]. However the OC-CCI data product that merges different chlorophyll data sources considerably increases the Red Sea coverage. However, this new dataset has not been used to revisit the assumption made in the large-scale Red Sea phytoplankton productivity. If for case I open water, the data in the Red Sea has been shown to be of quality comparable with  that the rest of the world [Brewin 2013], case II waters remain a challenge, especially in the SRS where the water is particularly shallow and the coral reef extended [Triant 2014, Raitsos 2013]. Theses problems generally lead to an overestimation of the chlorophyll level, but not necessarily to erroneous values [Raitsos 2013].

One of the most popular algorithm for the data filing of remotely-sensed data is DINEOF. It is an EOF based data filling approach introduced by [Beckers]. It has been used for multivariate reconstruction of SST fields using chlorophyll data in [Alvera 2007]. In [Sicarjobs 2011], it has been employed to fill chlorophyll data with 70\% of missing values. [Taylors 2013] has compared DINEOF with other EOF-based reconstruction algorithms and shown that the former is the best method for data filling. DINEOF has been employed in several other chlorophyll studies [miles 2007, Waite]

\section{Data Assimilation for marine ecology models}

Data assimilation schemes are used to improve the simulations of ecological dynamics models by correcting their predictions with observations. The most common use of data assimilation is to improve the forecast of an ecological model, by providing it with an estimated initial state. Such prediction capabilities are deployed in operational expert systems, for example to study the impact of human activities on the ecosystem of the Gulf of Pagasitikos [Korres 2012]. The deployment of such a forecasting system in the Red Sea is currently under study [Triant. 2014]. Hindcasting, the estimation of unobserved variables, is another application of assimilation scheme. [Ciavatta 2011] showed that he could improve the seasonal and annual hindcast of non assimilated biogeochemical properties in a shelf area (Wester English Channel). Finally, data assimilation can be used for reanalysis, to provide estimates of past years biogeochemical variables [Fontana 2013]. 

In the marine ecology modeling community, two assimilation schemes flavours have been widely used: Ensemble Kalman filters (EnKF) and the Singular Evolutive Extended Kalman filter (SEEK). The Stochastic EnKF, a Monte-Carlo approximation of the Kalman Filter, has been used in [Ciavatta 2011 and 2014]. However, it suffers from sampling errors when the ensemble size is smaller than the number of observations, as is usually the case when assimilating remotely-sensed data. The Singular Evolutive Interpolated Kalman filter (SEIK) is a deterministic version of the EnKF that do not suffer from sampling problems, as it projects the propagated error in a low-dimensional subspace. SEIK has been by [Trian 2012 and Korres 2012]. SEEK is a reduced order version of the Extended Kalman filter (EK), that in intractable in high-dimensions. Like SEIK, it projects the error covariance in a low dimensional space. SEEK has a long history in data assimilation for marine ecology models and is still used in recent studies [Fontana 2013, Korres 2012, Butenschon 2012]. [Korres 2012] shows that SEIK and SEEK are both comparably robust methods for highly non linear systems.

Current assimilation schemes are however affected by problems that have been addressed in the past years. First, biogeochemical variable are usually positive concentration, whereas Kalman filters expect Gaussian variables, and log-transformation can fail at solving this issue [Ciavatta 2011]. However, [Fontana 2013] has successfully introduced Gaussian anamorphosis transformations to solve this issue. Second, ecological blooms are intermittent and highly nonlinear, conditions that are challenging for assimilation schemes [Triant 2012, Korres 2012]. Third, SEIK and SEEK both project the error covariance in a subspace, resulting in an underestimation of the estimation error. [Butenschon 2012] studied different ways to propagate the error covariance in order to alleviate this issue. Finally, the model error statistics are required by Kalman-derived filters, but are difficult to estimate. [Triant 2012] proposes to use the $H_\infty$ method with SEIK in order remove this requirement.

\section{Statistical models for chlorophyll concentration}

Statistical and machine learning models have been used for estimation and classification problems related to phytoplankton concentrations. One application is the detection of harmful algal bloom from spatio-temporal satellite dataset, that has been addressed in [Gokaraju 2011] in the Gulf of Mexico using support vector machines. Another application is the estimation of chlorophyll concentration in case II coastal water using satellite radiance data. This problem has been addressed by [Kim 2014] on the west coast of South Korea, and by [Camps-Valls 2006] using a global dataset of in situ measurements. The former used the support vector regression algorithm, while the latter used also the random forest algorithm. 

Machine learning algorithms, in particular Artificial Neural Networks have been very popular for forecasting regional chlorophyll concentration in regions with very complex dynamics. In such regions, deterministic ecological models are usually too complicated to use and less efficient than data-driven approaches. Neural networks have been widely used for forecasting chlorophyll concentration in fresh as well as in coastal water systems. In [Jeong 2007], temporal recurrent recursive neural network have been used and found superior to traditional time-series model for daily forecasts of chlorophyll concentration. [Wang 2013] also used recurrent neural networks for daily chlorophyll forecasting in Lake Taihu, China. [Mulia 2013] combined Neural Network and genetic algorithm for nowcasting and forecasting of the chlorophyll concentration up to 14 days ahead, in the tidal dominated coast of Singapore. Finally, [Lee 2013] used neural networks for the forecasting of algal bloom with one and two weeks lags in the coastal waters of Hong-Kong. 

\section{Space-time geostatistical models for forecasting (Genton)}


