
% Thesis Introduction File

\chapter{Motivation}

\section{Importance of phytoplankton}

Phytoplankton are unicellular, free-floating, photosynthetic algae that live in the upper layers of bodies of water (ocean, lakes, rivers or ponds). There exists a wide diversity of phytoplankton species. To this date, about 5000 of them have been identified [Tepp 1995]. Phytoplankton are also highly variable in sizes, ranging from 0.2μm for cyanobacteria to 200μm for the largest species of diatom [Pal 2014]. In the oceans, phytoplankton live in the surface layer where there is enough sunlight for photosynthesis. 

Phytoplankton plays a fundamental role for the ocean ecology. It is the basis of the marine food web and traps most of the energy used by pelagic ecosystems [Pal 2014]. Zooplankton graze phytoplankton which are then consumed by higher trophic levels. It has been estimated that nearly 98\% of the ocean primary productivity comes from phytoplankton [Pal 2014]. Phytoplankton are also responsible for maintaining the dissolved oxygen level necessary for other species to survive. However, high phytoplankton concentration also impact their environment by creating dead zone when they die and are decomposed by bacteria, consuming all the available oxygen [Pal 2014]. Due to the rapid growth of phytoplankton, it responds very well to changes in its environment, making it a key parameter to monitor the water quality [Wu 2014].

Due to phytoplankton place at the bottom of the marine food chain, it is a key factor for fisheries. Productive fishing zones like the regions in the Arabian seas, Californian coast, north-west African coast and Chilean coast are explained by the upwelling of cold nutrient rich water favourable to phytoplankton, which in turn feeds higher trophic levels. On the other hand, during the El-Nino phenomenon creates less favourable conditions for phytoplankton in the Eastern Pacific, resulting in a dramatic reduction of fish catches of fisheries in the western coast of South America [Robison 2010]. Remotely-sensed chlorophyll data have been routinely used since the past decades to help fisheries predict the timing phytoplankton blooms [Robison 2010]. 

Phytoplankton also plays the role of a biological CO2 pump and strongly impact the Earth climate. During photosynthesis, phytoplankton captures carbon and releases oxygen. A part of this organic material stays in the food web, either transmitted to higher trophic level, or degraded  by bacteria. Another part, however, sinks to the bottom of the ocean and sediments. It is estimated that phytoplankton account for 48% of Earth carbon fixation [Pal 2014]. 
\section{Measuring Chlorophyll Concentration}

Chlorophyll is a molecule that is food in algae, phytoplankton and plants that is critical for photosynthesis. Phytoplankton is a poor absorber of green light, and is responsible for the coloration of plants. When phytoplankton are present in high concentrations they change the water also takes a detectable green coloration (it can also take a red or blue coloration depending on the type of phytoplankton dominating). This offers a way to estimate the chlorophyll concentration of the water, which is a good proxy for phytoplankton concentration. 

In-situ measurements of chlorophyll are however expensive and have limited temporal and spatial coverage. In-situ measurement of chlorophyll concentration can be gathered through scientific cruises, buoy stations or gliders (unmanned submarines). These methods are expensive to deploy and therefore the coverage is limited. Political issues, like in the Red Sea, can also be a practical barrier to in-situ measurements. 

Satellite measurements of chlorophyll provide excellent proxies for phytoplankton concentrations with a good temporal and spatial coverage. The SeaWIFS, MODIS and MERIS missions have provided an uninterrupted coverage of the world since 1997. High-resolution maps of daily chlorophyll concentration are freely accessible to the scientific community. Despite some limitations, in particular of missing data due to cloud coverage and sunglint, or problematic values in coastal areas, remotely-sensed chlorophyll concentration are used intensively by the scientific community. In regions, like in the Red Sea, where little in-situ measurements are available, it is often the most important data source. 

\section{Primary productivity in the Red Sea}

Typical tropical seas (TTS), like the Red Sea, are characterized by a highly stratified structure, where warm nutrient-depleted surface water is separated by cold nutrient-rich by a steep gradient of temperature zone called pycnocline. The pycnocline acts as barrier that prevents nutrients to reach the surface water. As a result, TTS are oligotrophic and have low chlorophyll concentrations. Until recently, marine biologists have thought that TTS had therefore a very low productivity. However, recent investigations have contested this idea, that different upwelling mechanisms exist that bring new nutrients to the surface water. 

Despite being an oligotrophic and challenging environments for marine life, the Red Sea presents a surprisingly rich and diverse ecosystem. Most of it lives in the very developed coral reef system. The source of nutrient for sustaining such a developed ecosystem is not well understood yet, but the interaction with the open sea through the mesoscale eddies is believed to play an important role. 

Remotely-sensed chlorophyll data show an important seasonality of the Red Sea primary productivity, that has been linked to winter deep mixing, and the inversion of the wind direction in the southern Red Sea, enhancing intrusion of nutrient rich Gulf of Aden water. Despite this strong seasonality, there is a large interannual variability caused by the unpredictable occurrence of large phytoplanktonic blooms. Diverse causes have been hypothesized for these blooms such as wind-induced mixing, eddies or dust storms carrying nutrients. 

Although the Red Sea environment is relatively preserved, it is under increasing pressure due to human activities. An abrupt increase of temperature has occurred in the last decade that threatens the fragile coral reef system. Moreover, the increasing urbanization and fishing activity contribute to the fragilization of this unique ecosystem.

\section{Chlorophyll Concentration Prediction}

Models can be useful to identify causes behind the chlorophyll patterns we observe in the Red Sea. Many hypotheses have be made about the drivers of chlorophyll concentration in this regions, but some of them have not been yet investigated through models. The role played by the exchange of water with the Gulf of Aden and winter overturning in the northern Red Sea have been successfully modeled with circulation and ecological models. However, the interaction between the open sea and coral reefs and the role of sand storms has not been investigated yet. Models, can also be helpful in discovering new dynamics affecting the chlorophyll concentration. In particular, the interaction between the productivity level of the different regions of the Red Sea has not been studied yet.  

Model predictions for chlorophyll concentration can also have practical applications. Phytoplankton blooms can be harmful to humans and marine life and are closely monitored in many regions of the world. In the Red Sea, where tourism and aquaculture are developing it is likely to become a concern too. Phytoplankton is also directly, and indirectly through zooplankton, the cause of microfouling that affects desalination plants. Anticipating a phytoplanktonic bloom might therefore be helpful in taking preventive actions. Finally, due to their short life-cycle, phytoplankton concentration reacts quickly to changes the environment, making it a  key variable in water quality monitoring.

\section{Modeling Chlorophyll Concentration}

Ecological ordinary differential equation (ODE) deterministic models are a popular way to model marine ecology. Such models can be as simple as the nutrient-phytoplankton-zooplankton (NPZ) model that only has three variables representing two trophic level, or as complex as the European regional seas ecosystem model (ERSEM) that has dozens of variables and represent many ecological, biological and chemical interactions. Such a model has been couple to the MITgcm circulation model used to simulate the Red Sea ecology. However the complexity of these models makes them difficult to deploy and interpret their results. 

On the other hand, data-driven statistical models are relatively easier to apply. They are relevant when the phenomenon producing the data is very complex or poorly known. They have been applied to predict chlorophyll concentration, mostly in small regions that have complex dynamics. Some statistical models, such as linear regression, GAM or tree regression have the advantage of being easy to interpret, and can be used to understand the dynamics driving the chlorophyll concentration.

Phenomena such as propagation and diffusion play a key role in the chlorophyll spatial concentration, but are difficult to represent without spatial modeling. There is also a difference in the chlorophyll patterns of different regions of the Red Sea, in particular between the nutrient rich southern Red Sea and the oligotrophic northern Red Sea, and between the open ocean and the coastal waters. There is however no clear cut division between regions with different pattern, making it difficult to divide the Red Sea into regions. Finally the different regions of the Red Sea are believed to interact. A model is therefore needed to account for the spatial and temporal interaction of the chlorophyll.

Classical geostatistics is the most widely used spatial statistical models. It models spatial data as the realization of a two dimensional Gaussian process, of which one can estimate the parameters. Geostatistics can be easily extended to spatio-temporal datasets. Many flexible ways of constructing space-time covariance functions for these models have been proposed recently. Space-time geostatistics has been applied to many environment studies, but not to chlorophyll data yet.

