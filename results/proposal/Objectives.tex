\chapter{Objectives}

The goal of the dissertation is to show that statistical predictive models can be used to help efficiently forecasting chlorophyll concentration in the Red Sea. Statistical methods can be more robust and computationally efficient when the underlying dynamics are very complex and observations are limited. The study of chlorophyll concentration is such a case. The dissertation will compare the 8-days prediction skill of increasingly sophisticated predictive models to the highly sophisticated ecological model ERSEM. We will explore the possibilities of combining statistical and deterministic models to improve the chlorophyll forecasts.

Efficient statistical models may be used as alternatives to the much more complex deterministic models in other seas. They can help researchers to gain insight into the dynamics of the phytoplankton. They can also help coastal communities to mitigate the effects of harmful phytoplankton blooms on public health and their economy. Moreover this study will help confirming hypothesizes that have been made about the interaction of Red Sea phytoplankton with the regional and global circulation. 
