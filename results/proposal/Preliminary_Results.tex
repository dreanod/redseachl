\chapter{Preliminary Results}

\section{Data Loading}

Most of the data necessary for the analysis has been gathered. Almost all of it
is freely available on Internet. Shell and R scripts have been written and used
for loading the following data all the data excepted the wind. The latter has
been provided by Yesubabu Viswanadhapalli, and is the result of the
assimilation of QuickScat satellite and in situ data to the WRF regional wind
model. There are additional datasets that could be interesting to use in the
analysis, but they are mainly climate mode time-series like IMI and EAWR, that
are easy to download. More details about the data can be found in table xxx in
the appendice.

So far, the MODIS and CCI data have been cropped over the region of interest,
cleaned and exported to the format TIFF, which can be read easily by most
software, R in particular. Each of these variables has then been aggregated in
a single file in the native R raster format. Applying this processing to the
remaining raster data should be straightforward. Then, the data will need to be
aligned and aggregated on the same temporal and spatial resolution, before
aggregating it in table format.

\section{Red Sea Chlorophyll Data Exploration}

The MODIS and CCI chlorophyll data products have been explored and compared.
The MODIS data presents an important amount of missing data that can reach
100\% during summer in the southern Red Sea, making any analysis in this region
very difficult. The CCI data product solve this issue by merging three sources
of remotely-sensed chlorophyll data (MODIS, MERIS and SeaWiFS) and using a new
algorithm for retrieving chlorophyll values when the cloud cover is important.
The increases the coverage to 70\% during summer month.

In a previous study (in appendice), the SeaWiFS chlorophyll data has been used.
The seasonal signal is the data is strong and has been shown to account for
50\% of the variability. The seasonal anomalies display a strong spatio
temporal correlation: the anomlaly at the same location from one week to the
other is correlated at 40\%, whereas two locations at 0.5 degrees apart are
nearly 60\% correlated. Not shown in this article, I also compared the SST and
chlorophyll data, and found an important negative correlation. However, when
looking at the anomalies, the correlation disappeared, suggesting that the
causes of seasonal and interannual variability are distinct.

\section{Red Sea Eco-Regions Clustering}

I used clustering algorithms in order to derive the Red Sea eco-regions. They
were applied to monthly log-concentration of chlorophyll. I used SeaWiFS data,
that has been previously filled using DINEOF. I used the popular k-means, and
the Gaussian Mixture Model (GMM) clustering algorithms.

I found that GMM provides more interesting results. With any number of
clusters, we obtain a division of the Red Sea into regions of comparable sizes.
With 5 clusters, the regions are similar to those defined in
\citet{Raitsos2013}. In addition to the purely latitudinal division proposed in
the former, we can guess the effects of the mesoscale eddies thanks the curved
division between regions. The fact that the curvature is oriented toward the
south might come from the fact that most nutrients propagate north from the
Gulf of Aden.

In Chapter 2 of the thesis, I plan to use the dataset constructed in Chapter 1.
By using CCI chlorophyll data instead of SeaWiFS, the need for data filling is
minimized. This is desirable, as data filling can introduce biases. It will
also be possible to use additional variables. For example, we can expect the
temperature and the bathymetry to have a large impact on the Red Sea
phytoplankton biology. Sea level anomaly can be useful in that it is indicates
the presence of mesoscale eddies. Finally, alternative clustering algorithms
will be tested.

\section{Global Geostatistical Model}

The research described in chapter 3 of the thesis has mostly been done and, is
the object of an article currently in review. The current version of the
manuscript can be found in appendice.

\section{Regional 1D Ecological Model}

The 1D regional ecological models used for this thesis have been configured
and are operational. Three models will be used: for the northern, central and 
southern Red Sea. The ecology is modeled with ERSEM, and the hydrodynamics
is modeled with MITgcm.

The results of MITgcm are those from \citet{Yao2014, Yao2014b}, in which a
simulation of the Red Sea and part of the Gulf of Aden circulation was run over
50 years. The NCEP data were used for atmospheric forcing, and the ECCO data
for the open boundary condition in the Gulf of Aden. These results are used for
the temperature and vertical circulation at the modeled points.

ERSEM simulates the complete water column with the pelagic and benthic
ecosystems, as well a their coupling. The equations model the flow of carbon,
nitrogen, phosphorus and silicon in the ecosystem. Living organisms are modeled
in terms of population processes (growth and mortality) and physiological
processes (ingestion, respiration, excretion, and egestion). The biota is
divided into functional groups according to their trophic levels: producers
(phytoplankton), consumers (zooplankton) and decomposers (bacteria), and
further subdivided according to their sizes.

The ecological models are initialized with the results of a 3D ecological
simulation of the Red Sea \citep{Triantafyllou2014}. The nutrient
concentrations however are initialized using values from the World Ocean Atlas
2005 (WOA 2005). These values are corrected by a factor chosen empirically.

\section{Data Assimilation for Ecological Models}

