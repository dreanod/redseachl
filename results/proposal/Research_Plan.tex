\chapter{Research Plan}

The thesis research plan including the different chapters is presented here. To
tackle the thesis objectives, I have organized the thesis in six chapters
organized in two major parts. In the first part, the data is collected and
analyzed, and then used to construct statistical models for chlorophyll
forecasting. The second part will present data assimilative ecological models,
and introduce innovative methods for enhancing the state of the art ensemble
methods, and merging statistical and deterministic models will be explored.

Chapter 1 presents and analysed the dataset that will be used in the following
chapters.  In chapter 2, we will divide the Red Sea into eco-regions using data
clustering techniques. Then, we will fit statistical models for forecasting
chlorophyll in these regions. In chapter 3, we will construct a spatio-temporal
geostatistical model for the global Red Sea chlorophyll, and use it for
forecasting. In chapter 4, this geostatistical model will be refined by using
several regional geostatistical models. In chapter 5, chlorophyll will be
forecasted using assimilative 1D regional deterministic ecological models, and
the results will be compared to the developed statistical models.  Chapter 6,
will develop a method to efficiently combine statistical and deterministic
models for improving chlorophyll forecasting in the Red Sea.

In the description of each chapter, I will first explain its importance for
reaching the thesis objectives. Then, I will list the open scientific questions
that will be addressed in the chapter. An outline of the methodology will be
also presented. Finally a list of expected outcomes, will be presented.
