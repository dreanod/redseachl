\chapter{Research Plan}

The research plan including the different chapters of the thesis is presented
hereafter. To tackle the objectives, I have organized the thesis in six
chapters composed of the first four chapters in which the data is collected and
analyzed (chapter 1), and used to construct statistical (chapter 2) and
geostatistical models (chapter 3 and 4) that are used for chlorophyll
forecasting. Data assimilative ecological models are then introduced in chapter
5. Innovative methods for enhancing the state of the art ensemble methods will
be explored in chapter 6.

For each chapter, I will first explain its importance for reaching the thesis
objectives. Then, a list of open scientific questions will be presented, that
the chapter will contribute to. An outline of the methodology will be
introduce. Finally a list of outcomes, will be given against which the chapter
can be evaluated.

Chapter 1 presents the dataset that will be used in the following chapters.  In
chapter 2, we will divide the Red Sea into eco-regions using clustering
algorithms. Then, we will fit statistical models for forecasting chlorophyll in
these regions. In chapter 3, we will construct a spatio-temporal geostatistical
model for the global Red Sea chlorophyll, and use it for forecasting. In
chapter 4, this geostatistical model will be refined by using several regional
geostatistical models. In chapter 5, chlorophyll will be forecasted using
assimilated 1D regional deterministic ecological models, and the results will
be compared to the statistical models.  Chapter 6, will develop a method to
combine statistical and deterministic models for improving chlorophyll
forecasting in the Red Sea.
