\chapter{Research Plan}

The thesis research plan including a description of the thesis chapters is
presented here. To tackle the thesis objectives, I have organized the thesis in
six chapters that could be partitioned in two major parts. In the first part,
the data is collected and analyzed, and then used to construct statistical
models for chlorophyll forecasting. The second part will present data
assimilative ecological models, and introduce innovative methods for enhancing
the state-of-the-art ensemble assimilation methods. I will also explore and
propose ways to combine statistical and
deterministic models for efficient forecasting of chlorophyll in the Red Sea.

Chapter 1 presents and analyses the dataset that will be used in the following
chapters.  In chapter 2, I will divide the Red Sea into eco-regions using data
clustering techniques. Then, I will construct local statistical models for
forecasting chlorophyll in these regions. Chapter 3 will propose a
spatio-temporal geostatistical model for simulating the global Red Sea
chlorophyll, and use it for forecasting. In chapter 4, this geostatistical
model will be refined by using several regional geostatistical models. In
chapter 5, assimilative 1D regional deterministic ecological models will be
configured and implemented, and the results will be compared to the developed
statistical models.  Chapter 6, will develop a method to efficiently combine
statistical and deterministic models for improving chlorophyll forecasting in
the Red Sea.

In the description of each chapter, I will first explain its importance for
addressing the thesis objectives. Then, I will list the open scientific questions
that will be addressed in the chapter. An outline of the methodology will be
also presented. The chapter description will conclude with a list of expected
outcomes.
