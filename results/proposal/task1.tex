\section{Chapter 1: Dataset Building and Exploration}

A preliminary task to data modeling, is the gathering, cleaning and exploration
of the data. Given the complexity and the size (40 GB) of the dataset, this is not
an easy task. This first data analysis, will reveal if enough data has been
gathered to make meaningful forecast, and what accuracy we can expect from the
models. This step will also provide information that will help in designing
statistical models: most significant variables, differences between regions,
relevant data transformation, etc. Finally, this step will identify patterns in
the data that will be useful to qualitatively evaluate predictive models.

\paragraph{Open Questions}

\begin{itemize} 

\item Can we efficiently identify outliers in the chlorophyll values?

\item Is there a way to efficiently fill the missing values in the chlorophyll
dataset?

\item Can the data help understanding the mechanisms behind extreme blooms in
the Red Sea?

\item Will the data support the hypothesizes about the dynamics behind the
chlorophyll seasonal cycle in the Red Sea?

\end{itemize}

\paragraph{Methods}

\begin{enumerate} 

\item Identify data sources and gather the data.

\item Clean the data and fill in missing values (DINEOF).

\item Align and format the data to build a cleaned dataset.

\item Explore the dataset.

\begin{itemize} 

\item Study the correlation between chlorophyll and other variables (Linear
Regression, GAM, data transformations) in order to know which ones will
be important in the model.

\item Do variable selection (Lasso, single variable regression, multistep
regression): as too many variables may introduce noise in the models or make
the fitting slower.

\item Study the regional aggregation (ACF), to characterize the 
clusters.

\item Explore spatiotemporal correlations (hovmoller plots, PCA, variograms)
to get an understanding on the spatial features of the data.

\end{itemize}

\end{enumerate}

\paragraph{Expected Outcomes}

\begin{itemize} 

\item A cleaned dataset that can be used in the following tasks.

\item A comprehensive exploration of the available data to study chlorophyll
variability in the Red Sea.

\item A preliminary variable selection.

\item A clear picture of the major spatio-temporal patterns in the data.

\item A critical evaluation of the current hypotheses about the chlorophyll
dynamics in the Red Sea.

\end{itemize}
