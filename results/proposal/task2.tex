\section{Chapter 2: Forecasting Chlorophyll Concentration in Regional
Aggregates}

The complexity of marine ecosystem is reflected in that of chlorophyll data. 
It is therefore useful to first simplify it
by aggregating it spatially. The space-time dynamics of the chlorophyll data
reflects the highly nonlinear dynamics of the underlying physical, chemical and
biological phenomenons. As shown by the north-south gradient and the seasonal
behavior, the resulting space-time process is nonstationary in time and in
space. The high-dimensionality in space can be reduced by considering a
regional aggregation of the data. This would allow us to focus on the global
scale phenomenons: such as the interactions between neighboring regions, the
time-scale of large events and the difference in the physical variables
affecting the chlorophyll concentration in each region. In the following tasks,
these simple predictive models will also be a reference for evaluating more
complex ones. 

\paragraph{Open Questions}

\begin{itemize}

\item Is the biological aggregation of the Red Sea proposed in
\cite{Raitsos2013} statistically meaningful?

\item Can clustering methods be used to identify marine ecological zones based
on chlorophyll data?

\item Would a simple forecasting model allow us to identify the causes of
chlorophyll blooms?

\end{itemize}

\paragraph{Methods}

\begin{enumerate}

\item Define the datasets on which the models will be trained and tested,
as well as the cross-validation methodology.

\item Apply unsupervised learning algorithms to the dataset to divide
the Red Sea into clusters corresponding to different environmental
conditions.

\item Build regional statistical models to forecast chlorophyll concentration.

\item Evaluate the models on the prediction of extreme blooms.

\end{enumerate}

\paragraph{Expected Outcomes}

\begin{itemize}

\item A division of the Red Sea into environmentaly distinct regions that has
been quantitatively evaluated.

\item A critical evaluation of current hypotheses about the chlorophyll
dynamics in the Red Sea.

\item A lower bound for the performance of a more sophisticated model.

\item An assessment of the limitation of aggregate methods for Chlorophyll
data.

\item An understanding of how the treatment of spatial correlations may improve
cholorphyll forecasting.

\end{itemize}
