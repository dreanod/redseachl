\section{Chapter 2: Forecasting Chlorophyll Concentration in Regional Aggregates}

\noindent
\emph{Duration: 2 months (by February 2015)}

\noindent
\emph{Submission: Progress in Oceanography}

\noindent
\emph{Collaborator: Dionysios Raitsos}

Chlorophyll data is very complex. It is therefore useful to first simplify it by aggregating it spatially. The space-time dynamics of the chlorophyll data reflects the highly nonlinear dynamics of the underlying physical, chemical and biological phenomenons. As shown by the north-south gradient and the seasonal behavior, the resulting space-time process is nonstationary in time and in space. The high-dimensionality in space can be reduced by considering a regional aggregation of the results. This would allow us to focus on the global scale phenomenons: such as the interactions between neighboring regions, the time-scale of large events and the difference in the physical variables affecting the chlorophyll concentration in each region. In the following tasks, these simple predictive models will also be a reference for evaluating more complex ones. 

\paragraph{Open Questions}

\begin{itemize}
\item Is the biological aggregation of the Red Sea proposed by (Raitsos 2013) statistically meaningful?
\item Can clustering methods be used to identify marine ecological zones based on chlorophyll data?
\item Can a simple forecasting model allow us to understand the causes of chlorophyll blooms?
\item Can the current hypothesizes about the seasonal chlorophyll dynamics be validated?
\end{itemize}

\paragraph{Method}

\begin{enumerate}
\item Define datasets (training and test datasets, cross-validation).....................................0%
\item Variable selection (Lasso, L1 regression, single-variable linear regression)...............0%
\item Define regional aggregations (unsupervised learning, Hierarchical clustering, K-means)...................................................................................................................50%
\item Forecasts chlorophyll concentration (linear regression, GAM models, diagnostic, k-nearest neighbors)....................................................................................................0%
\item Predicting future extreme blooms (nearest-neighbours, logistic regression, decision trees)...........................................................................................................................0%
\end{enumerate}

\paragraph{Expected Outcomes}

\begin{itemize}
\item A regional division of the Red Sea that has been quantitatively evaluated.
\item A critic of current hypothesis about the chlorophyll dynamics in the Red Sea.
\item A lower bound on the performance of a more sophisticated model.
\item An assessment of the limitation of aggregate methods for Chlorophyll data.
\item An understanding on how the treatment of spatial correlations can improve the results.
\end{itemize}

\paragraph{Work Accomplished and Prelimnary results}
