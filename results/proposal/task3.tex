\section{Chapter 3: Global Geostatistical Model for Chlorophyll Forecasting}

\noindent
\emph{Duration: 1 month (by March 2015)}

\noindent
\emph{Submission: Spatial Statistics}

Geostatistical methods can be used to construct dynamical models for forecasting the chlorophyll concentrations that we can compare to deterministic models. Geostatistics is a robust method to model spatio-temporal data. Recently there has been a lot of interest for expanding it to model spatio-temporal data. Through the use of Kriging interpolation, these models provide powerful tools for do spatio-temporal prediction. As a particular case of Kriging, by predicting the spatial future field given the observation of the present field, we can derive a linear dynamical model. This linear model can be employed in a filtering setting like the Kalman filter. This is a desirable framework, and is similar to the way deterministic models are employed to make forecasts given past observations. 

\paragraph{Open Questions}

\begin{itemize}
  \item Can a global geostatistical model fit chlorophyll data?
  \item How non stationary is the data in time and space?
  \item What spatiotemporal covariance functions best fit the chlorophyll data?
  \item Can geostatistical methods be employed in a filtering setup?
\end{itemize}

\paragraph{Method}

This task has already been started and had been the object of a submission for publication. The remaining work includes:
\begin{itemize}
  \item Use the new dataset and the new covariates
  \item Compare the results to the ones of with the regional aggregates
\end{itemize}

\paragraph{Expected Outcomes}

\begin{itemize}
  \item A methodology to employ a geostatistical model in a filtering problem.
  \item A characterization of the space-time non stationarity of the data, and the interaction of the temporal and spatial dimensions.
  \item An understanding of how spatial aggregation and geostatistical models can be used in the same model. 
\end{itemize}

\paragraph{Work Accomplished and Preliminary Results}


