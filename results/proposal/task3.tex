\section{Chapter 3: A Global Geostatistical Model for Chlorophyll Forecasting}

Geostatistical methods can be used to construct dynamical models for
forecasting the chlorophyll concentrations. Geostatistics is a robust approach to
model spatial data.  Recently there has been increasing interest for expanding
it to model spatio-temporal data. Through the use of Kriging interpolation,
these models provide powerful tools for spatio-temporal prediction. As a
particular case of Kriging, by predicting the spatial future field given the
observation of the present field, one can derive a linear dynamical model for
forecasting the data. This linear model may be employed in a filtering setting
like the Kalman filter. This is a desirable framework, and is similar to the
way deterministic models are employed to compute forecasts given past
observations. 

\paragraph{Open Questions}

\begin{itemize}

\item Can we construct a global geostatistical model that fits well enought
chlorophyll data in the Red Sea?

\item How non-stationary the chlorophyll data is in time and space?

\item Which spatio-temporal covariance functions best fit the chlorophyll concentration?

\item Can geostatistical methods be employed in a filtering setup?

\end{itemize}

\paragraph{Method} 
\mbox{}

Most of the work proposed in this chapter has already been accomplished and has
been the subject of a paper that is currently under minor revision in Spatial 
Statistics.

\paragraph{Expected Outcomes}

\begin{itemize}

\item A new methodology to employ a geostatistical model in a filtering setup.

\item A characterization of the space-time non stationarity of the data.

\item A framework for using spatial aggregation and geostatistical models in
the same model.

\end{itemize}
