\section{Chapter 4: Local Geostatistical Model for Chlorophyll Forecasting}

\noindent
\emph{Duration: 3 months (by June 2015)}

\noindent
\emph{Submission: Journal of the American Statistical Association (Case Study)}

\noindent
\emph{Collaborator: Raphael Huser}

This part will bring together the results of the two preceding tasks to develop a predictive model that takes into account the large-scale dynamics and the regional spatio-temporal dynamics. In task 2, a predictive model is built, that represents the large scales behaviour of the Red Sea, but the spatial dimension inside each region is not addressed. We expect local features to play a role, such as the proximity to the coast, the bathymetry, proximity to other regions or major cities, etc. In task 3, we developed a methodology to use a geostatistical model in a dynamic fashion to do pixel-scale forecast. In this task, each regions will be modeled separately by a local geostatistical model that can do local prediction. These models will have access to aggregate covariates from neighboring regions in represent the global scale behaviours. 

\paragraph{Open Questions}

\begin{itemize}
\item What are the most adapted space-time covariance models for chlorophyll data?
\item How to use global covariates in a geostatistical model?
\item What are the differences in the fine-scale dynamics of chlorophyll in each region?
\item Can the fine scale behaviour of phytoplankton be predicted accurately?
\item What are the spatial features that are important for the chlorophyll dynamics?
\end{itemize}

\paragraph{Method}

\begin{itemize}
\item Extract local dataset from previous tasks
\item Design the training and test datasets, and the cross-validation method
\item Design and evaluate the mean function given the past covariates
\item Fit the local geostatistical model to the residuals.
\item Evaluate the model predictions and compare the results with task 2 and 3.
\end{itemize}

\paragraph{Expected Outcomes}

\begin{itemize}
\item A methodology to aggregate local geostatistical models
\item An improvements in the prediction skills over the models of task 2 and 3.
\item An understanding of the differences between each regions.
\item A critical evaluation of the space-time covariance models for fitting chlorophyll data.
\item A better characterization of the regional chlorophyll dynamics.
\end{itemize}

\paragraph{Work Accomplished and Preliminary Results}

