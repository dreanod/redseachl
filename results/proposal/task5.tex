\section{Chapter 5: Assimilation of Regional 1D Ecological Models and
Comparison to Statistical Models}

The three previous chapters focus on constructing increasingly sophisticated
statistical predictive models for the chlorophyll concentration in the Red Sea.
In this chapter these models will be evaluated against a 1D ecological model
(ERSEM). The goal of this part will be to identify the merits of each modeling
approach, and propose ways in which they can complement each other.  The models
will be run in each of the regions found in chapter 2.  Available data will
also be assimilated to the model through an ensemble Kalman-based smoothing
assimilation scheme that we will further equip with an efficient
expectation-maximization algorithm for parameters estimation.

\paragraph{Open Questions}

\begin{itemize}

\item Are statistical methods competitive for forecasting chlorophyll
concentrations?

\item How may statistical and deterministic models complement each other?

\item May statistical methods forecast interesting dynamical features?

\item Are there significant regional differences in the relative performances
of both approaches?

\item How best to estimate the parameters of ecological models? 

\end{itemize}

\paragraph{Methods}

\begin{enumerate}

\item Calibrate the ERSEM model on each of the regions to have useful
regional models running.

\item Define an assimilation scheme and the data that will be assimilated
by each regional model.

\item Implement the assimilation scheme.

\item Run the simulation and aggregate the results to allow for comparison
with the statistical models introduced in the previous chapters.

\item Define the metrics for comparison between statistical and deterministic
models.

\item Conduct the comparison with the statistical models and summarize
the results.

\end{enumerate}

\paragraph{Expected Outcomes}

\begin{itemize}

\item A complete set of measures of the prediction skills of each approach.

\item A method to estimate parameters in an assimilative ecological model.

\item A set of case studies of the behaviours of each method for forecasting
blooming events.

\item An understanding of the limitations of geostatistical models to predict
nonlinear dynamics. 

\item Propositions on how the statistical and dynamical approaches can
complement each other.

\end{itemize}
