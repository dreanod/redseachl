\section{Chapter 5: Assimilation of 1D ecological models and comparison to statistical models}

\noindent
\emph{Duration: 3 months (by September 2015)}

\noindent
\emph{Submission: Journal of Geophysical Research}

\noindent
\emph{Collaborator: George Triantafyllou / Boujeema}

The three previous tasks focus on constructing increasingly sophisticated predictive models for the chlorophyll concentration in the Red Sea. In this part these models will be compared to a 1D ecological model (ERSEM). This model is well detailed and very complex. The goal of this part will be to identify the merits of each modeling approach, and propose ways in which they can complement each other. To allow for comparison, the model will be run in each of the regions found in task 2. Available data will also be assimilated to the model through a smoothing assimilation scheme that will use an expectation-maximization algorithm for parameters estimation. 

\paragraph{Open Questions}

\begin{itemize}
\item Are statistical methods competitive for forecasting chlorophyll concentrations?
\item How can statistical and deterministic models complements each other?
\item Can statistical method forecast interesting dynamical features?
\item Are there significant regional differences in the relative performances of both approaches?
\item How to estimate the parameters of ecological models? 
\end{itemize}

\paragraph{Method}

\begin{enumerate}
\item Define the metrics for comparison
\item Calibrate the ERSEM model on each of the regions
\item Define an assimilation scheme and the data for the ERSEM model
\item Implement the assimilation scheme
\item Run the simulation and aggregate the results
\item Do the comparisons with the statistical models
\end{enumerate}

\paragraph{Expected Outcomes}

\begin{itemize}
\item A complete set of measures of the prediction skills of each approach.
\item A method to estimate assimilation and model parameters in an assimilated ecological model.
\item A set of case studies of the behaviours of each method for forecasting interesting events.
\item An understanding of the limitations of geostatistical models to predict nonlinear dynamics. 
\item Propositions on how the two approaches can complement each other.
\end{itemize}

\paragraph{Work Accomplished and Preliminary Results}

