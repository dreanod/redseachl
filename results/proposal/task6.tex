\section{Task 6: Improving an Ecological Model Data Assimilation Scheme through Statistical Predictive Models}

\noindent
\emph{Duration: 3 months (by September 2015)}

\noindent
\emph{Submission: Journal of Geophysical Research}

\noindent
\emph{Collaborator: George Triantafyllou / Boujeema}

\section{Motivation}

In the previous task, we compared the forecasts of the ecological ERSEM model to that of the statistical models we developed from tasks 2 to 4. In this task we will study how these two approaches can be complementary. Specifically, we will study the use of statistical forecasts model to improve the forecasts of the ERSEM ecological model. The forecasts of the statistical models will be treated as observations, that can be assimilated by the filtering scheme used with the ERSEM model, and will give an improved forecast. When real observations will be available, they will be assimilated sequentially. This, method will allow the different ERSEM models on each cluster to communicate indirectly their states to one another. 

\subsection{Open Questions}

\begin{itemize}
\item Can statistical predictive models be used to communicate information between deterministic model?
\item Would the access to information about other regions improve the model forecasts?
\item What are the global patterns of ecological dynamics in the Red Sea?
\end{itemize}

\subsection{Method}

\begin{enumerate}
\item Define new assimilation scheme
\item Define metrics to measure model improvement
\item Prepare training and test datasets
\item Train statistical model
\item Run simulation with assimilation of statistical observation
\item Compare with results of task 5
\end{enumerate}

\subsection{Expected Outcomes}

\begin{itemize}
\item An improvement in the prediction skills of the deterministic approach
\item A methodology to couple deterministic ecological models through statistical models
\item Insights on the global ecological dynamics of the Red Sea
\end{itemize}

