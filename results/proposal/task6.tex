\section{Chapter 6: Combining Statistical and Data Assimilative Predictive
Models for Improved Chlorophyll Forecasting}

In the previous chapters, we proposed to evaluate the forecasts of the
ecological ERSEM model with that of the statistical models we proposed from
chapters 2 to 4. In this chapter we will study how these two approaches can be
complementary. Specifically, we will study the use of statistical forecasts
model to improve the forecasts of the ERSEM ecological model. The forecasts of
the statistical models will be treated as observations, that can be assimilated
by the filtering scheme used with the ERSEM model, which will hopefully lead to
improved forecasts. On the other hand, real observations will be assimilated
sequentially when they become available. This method will allow the different
ERSEM models on each cluster to communicate indirectly their states to one
another. 

\paragraph{Open Questions}

\begin{itemize}

\item Can statistical predictive models be used to communicate information
between deterministic models?

\item Would the access to information about other regions improve the model
forecasts?

\item What are the global patterns of ecological dynamics in the Red Sea?

\end{itemize}

\paragraph{Methods}

\begin{enumerate}

\item Develop a new assimilation scheme to assimilate statistical predictions.

\item Define metrics to measure model improvement.

\item Prepare training and test datasets.

\item Train statistical model to predict chlorophyll.

\item Run simulations with assimilation of statistical and real observations.

\item Compare with results of task 5.

\end{enumerate}

\paragraph{Expected Outcomes}

\begin{itemize}

\item An improvement in the prediction skills of the dynamically-driven
approach.

\item A methodology to couple assimilative dynamically-driven regional
ecological models through statistical models.

\item Insights on the global ecological dynamics of the Red Sea.

\end{itemize}
